\documentclass[11pt]{article}

\usepackage[utf8x]{inputenc}
\usepackage[L7x]{fontenc}  
\usepackage[pdftex]{graphicx}
\usepackage{fourier}
\usepackage{amssymb}
\usepackage{amsmath}
\usepackage{wrapfig}
\usepackage{sectsty}
\usepackage{asymptote}
\usepackage{colonequals}
\usepackage{color}
\usepackage{calc}
\usepackage{etoolbox}

% COMMENT OUT THE SECOND LINE TO MAKE A HANDOUT WITHOUT SOLUTIONS
\newtoggle{solutions}
\toggletrue{solutions}

\usepackage{amsthm}
\theoremstyle{definition}
\newtheorem{theorem}{Theorem}
\newtheorem{defn}{Definition}
\newtheorem{lemma}{Lemma}
\newtheorem{corollary}{Corollary}
\newtheorem{exercise}{Exercise}

\sectionfont{\large}
\sectionfont{\normalsize}

\def\Arg{\mathop{\rm Arg}\nolimits}
\def\Res{\mathop{\rm Res}}
\renewcommand\Im{\mathop{\rm Im}\nolimits}
\renewcommand\Re{\mathop{\rm Re}\nolimits}
\newcommand\Arctan{\mathop{\rm Arctan}\nolimits}
\newcommand{\R}{\mathbb{R}}
\newcommand{\C}{\mathbb{C}}
\newcommand{\N}{\mathbb{N}}
\newcommand{\Z}{\mathbb{Z}}
\renewcommand{\P}{\mathbb{P}}
\newcommand{\Chat}{\hat{\mathbb{C}}}
\newcommand{\UHP}{\mathbb{H}}
\DeclareMathOperator{\area}{area}
\DeclareMathOperator{\dist}{dist}
\DeclareMathOperator{\interior}{int}
\DeclareMathOperator{\id}{id}

\pagestyle{empty}

\textwidth = 6.5 in
\textheight = 9 in
\oddsidemargin = -0.125 in
\evensidemargin = 0.0 in
\topmargin = -0.2 in
\headheight = 0.0 in
\headsep = 0.2 in
\parskip = 0.2 in
\parindent = 0.0 in

\def\inv{^{-1}}

\newcounter{prob}
	\setcounter{prob}{1}

\newcounter{subprob}
	\setcounter{subprob}{1}

\newcommand\itm{\theprob.  \stepcounter{prob}\setcounter{subprob}{1}}
\newcommand\subitm{(\alph{subprob}) \refstepcounter{subprob}}

\newcommand\sol[2]{\iftoggle{solutions}{\textit{Solution}. #1}{#2}}
%\newcommand\sol[2]{#2}


\newcommand{\problem}[1]{
\makebox[0.2cm]{\textbf{\itm}}  \begin{minipage}[t]{\linewidth-0.75cm}
#1
\end{minipage}
}

\newcommand\twomatrix[4]{
\left(
\begin{array}{cc}
#1 & #2 \\
#3 & #4
\end{array}
\right)
}

\renewcommand\vec[1]{\mathbf{#1}}
\newcommand\pd[2]{\frac{\partial #1}{\partial #2}}

\begin{document}
\thispagestyle{empty}

\begin{center}
  18.022 Recitation Handout \iftoggle{solutions}{(with solutions)}{} \\
  15 October 2014 
\end{center}


\itm Sketch the image of the path $\vec{x}(t)=(\cos t, \sin 2t)$.

\sol{
As $t$ goes from $0$ to $\pi/2$, the $x$-coordinate of $\vec{x}$ varies
from 1 to 0, while the $y$-coordinate varies from 0 to 1 and back to
0. Superimposing these two ``pen'' movements (like an etch-a-sketch), we get a
bump going from $(1,0)$ to $(0,0)$, as shown in the first quadrant
below. Letting $t$ continue to increase produces the three more copies of this
shape in the other three quadrants. 
\begin{center}
  \includegraphics{figures/parametric4} 
\end{center}
}{

\vfill
}

\itm Find the arclength of the graph of $f(x)=\frac23(x-1)^{3/2}$ between the points $(1,0)$ and $(4,2\sqrt{3})$. 

\sol{
We calculate $\int_{1}^4 \sqrt{1+f'(x)^2}=\int_{1}^4
\sqrt{1+(\sqrt{x-1})^2}\,dx=\boxed{14/3}$. 
}
{
\vfill
}

\iftoggle{solutions}{}{\newpage}

\itm Consider the function $\mathbf{F}:\R^3\to\R^2$ defined by $\mathbf{F}(x,y,z)=(3x/y,2x+e^z)$.

(a) Find $D\mathbf{F}$.

\sol{ The total derivative is the matrix of partial derivatives:
\[
\left(
  \begin{array}{ccc}
    3/y & -3x/y^2 & 0 \\
    2 & 0 & e^z
  \end{array}
\right)
\]
}
{\vfill}

(b) Show that there exists an open set $U\subset \R$ containing $1$ and a function $\mathbf{f}:U\to \R^2$ such that for all $x\in U$, the equations 
$\mathbf{F}(x,y,z)=\mathbf{F}(1,-2,0)$
have a unique solution $(y,z)=\mathbf{f}(x)$. Show that $\mathbf{f}$ is $C^1$. 

\sol{ The implicit function theorem ensures that we can solve (abstractly)
  for $(y,z)$ in terms of $x$ if the matrix of partial derivatives
  corresponding to the $y$ and $z$ columns has nonvanishing determinant. In
  this case, that means 
\[
\det \left(
  \begin{array}{ccc}
    -3x/y^2 & 0 \\
    0 & e^z
  \end{array}
\right) = \det \left(
  \begin{array}{cc}
    -3/4 & 0 \\
    0 & 1
  \end{array}
\right) = -3/4 \neq 0,
\]
so the implicit function theorem does apply and gives us the desired
function $\vec{f}$. The theorem also tells us that $\vec{f}$ is $C^1$. 
} {\vfill }

(c) Find $D\mathbf{f}(1)$. 

\sol{
  Since $\vec{F}(x,\vec{f}(x))=\vec{c}$ for some constant $\vec{c}$, we can
  differentiate both sides to get $D\mathbf{f}=-A^{-1}B$, where the matrix
  $A$ is the one obtained by taking the $y$ and $z$ columns of
  $D\vec{F}$ and $B$ is the matrix obtained by considering the remaining
  columns. We get 
\[
D\mathbf{f}(1) = \left(
  \begin{array}{cc}
    -3/4 & 0 \\
    0 & 1
  \end{array}
\right)^{-1}\left(
  \begin{array}{c}
    -3/2 \\
    0 
  \end{array}
\right) = \boxed{\left(\begin{array}{c}-2 \\-2\end{array}\right)}
\]
}{
\vfill
}


\itm (from 3.2.15 in \textit{Colley}) Determine the moving frame $\{\vec{T},\vec{N},\vec{B}\}$, the curvature, the torsion, and the arc length parameter $s(t)$ for the curve 
\[ 
\vec{x} = \left(5, \frac13(t+1)^{3/2}, \frac13(1-t)^{3/2}\right), \qquad -1 < t < 1.  
\]


\sol{We have 
\begin{align*}
\vec{T} &= \vec{x}'(t)/\|\vec{x}'(t)\| = \left(1,\frac12\sqrt{t+1},-\frac12\sqrt{1-t}\right)/\sqrt{3/2}, \\
\vec{N} &= \frac{d\vec{T}/dt}{\|d\vec{T}/dt\|} = 2\sqrt{2}\left(0,\frac12 \sqrt{1-t},\frac14\sqrt{t+1}\right), \\
\vec{B} &=\vec{T}\times \vec{N} = (1,-\sqrt{t+1},\sqrt{1-t})/\sqrt{3}.
\end{align*}
The curvature and torsion are $\dfrac{\sqrt{2}}{6(1-t^2)}$ and $\dfrac{1}{3\sqrt{1-t^2}}$, respectively. The arclength parameter is $s(t) = \int_{0}^{t} \|\vec{x}'(\tau)\|\,d\tau = t/\sqrt{2}$. 
}{\vfill}


\end{document}
