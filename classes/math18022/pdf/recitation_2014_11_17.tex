\documentclass[11pt]{article}

\usepackage[utf8x]{inputenc}
\usepackage[T1]{fontenc}  
\usepackage[pdftex]{graphicx}
\usepackage{hyperref}
\usepackage{amsmath}
\usepackage{amssymb}
\usepackage{wrapfig}
\usepackage{sectsty}
\usepackage{colonequals}
\usepackage{color}
\usepackage{calc}
\usepackage{etoolbox}

% COMMENT OUT THE SECOND LINE TO MAKE A HANDOUT WITHOUT SOLUTIONS
\newtoggle{solutions}
%\toggletrue{solutions}

\usepackage{amsthm}
\theoremstyle{definition}
\newtheorem{theorem}{Theorem}
\newtheorem{defn}{Definition}
\newtheorem{lemma}{Lemma}
\newtheorem{corollary}{Corollary}
\newtheorem{exercise}{Exercise}

\usepackage{palatino} 
\usepackage{pxfonts} 

\sectionfont{\large}
\sectionfont{\normalsize}

\def\Arg{\mathop{\rm Arg}\nolimits}
\def\Res{\mathop{\rm Res}}
\renewcommand\Im{\mathop{\rm Im}\nolimits}
\renewcommand\Re{\mathop{\rm Re}\nolimits}
\newcommand\Arctan{\mathop{\rm Arctan}\nolimits}
\newcommand{\R}{\mathbb{R}}
\newcommand{\C}{\mathbb{C}}
\newcommand{\N}{\mathbb{N}}
\newcommand{\Z}{\mathbb{Z}}
\renewcommand{\P}{\mathbb{P}}
\newcommand{\Chat}{\hat{\mathbb{C}}}
\newcommand{\UHP}{\mathbb{H}}
\DeclareMathOperator{\area}{area}
\DeclareMathOperator{\dist}{dist}
\DeclareMathOperator{\interior}{int}
\DeclareMathOperator{\id}{id}

\pagestyle{empty}

\textwidth = 6.5 in
\textheight = 9 in
\oddsidemargin = -0.125 in
\evensidemargin = 0.0 in
\topmargin = -0.2 in
\headheight = 0.0 in
\headsep = 0.2 in
\parskip = 0.2 in
\parindent = 0.0 in

\def\inv{^{-1}}

\newcounter{prob}
	\setcounter{prob}{1}

\newcounter{subprob}
	\setcounter{subprob}{1}

\newcommand\itm{\theprob.  \stepcounter{prob}\setcounter{subprob}{1}}
\newcommand\subitm{(\alph{subprob}) \refstepcounter{subprob}}

\newcommand\sol[2]{\iftoggle{solutions}{\begin{proof}[Solution] #1\end{proof}}{#2}}
%\newcommand\sol[2]{\iftoggle{solutions}{\textit{Solution}. #1}{#2}}
%\newcommand\sol[2]{#2}


\newcommand{\problem}[1]{
\makebox[0.2cm]{\textbf{\itm}}  \begin{minipage}[t]{\linewidth-0.75cm}
#1
\end{minipage}
}

\newcommand\twomatrix[4]{
\left(
\begin{array}{cc}
#1 & #2 \\
#3 & #4
\end{array}
\right)
}

\renewcommand\vec[1]{\mathbf{#1}}
\newcommand\pd[2]{\frac{\partial #1}{\partial #2}}

\begin{document}
\thispagestyle{empty}

\begin{center}
  18.022 Recitation Handout \iftoggle{solutions}{(with solutions)}{} \\
  17 November 2014 \\
\end{center}

\itm Consider the surface $S=\left\{(x,y,z)\in \R^3\,:\,x>0\text{ and }r = 1\text{ and }-\sqrt{\frac{\pi^2}{4}-\theta^2}\leq z\leq \sqrt{\frac{\pi^2}{4}-\theta^2}\right\}$, shown below. (Note that $r$ and $\theta$ refer to cylindrical coordinates.) 

(a) Find the surface area of $S$ using a scalar line integral. 

(b) Check your answer by finding a non-calculus method of calculating the area of $S$. 

\begin{center}
  \includegraphics{figures/label}
\end{center}

\sol{(a) We integrate $f(x,y) = 2\sqrt{\frac{\pi^2}{4}-\theta^2}$ along the semicircular arc $C$ in the $xy$-plane from $(0,-1,0)$ to $(0,1,0)$. We note that for the path $\vec{x}(\theta)=(\cos\theta,\sin\theta,0)$, we have $\|\vec{x}'(\theta)\|d\theta = d\theta$. So 
\begin{align*}
  \int_C f \,ds &=\int_{-\pi/2}^{\pi/2} 2\sqrt{\frac{\pi^2}{4} -\theta^2}\,d\theta = \boxed{\pi^3/4}. \\
\end{align*}
(b) $S$ is just a disk wrapped around a cylinder. The area of the circle is $\pi r^2 = \pi(\pi/2)^2=\pi^3/4$. 
}{\vfill}


\iftoggle{solutions}{}{\newpage}

\itm In this problem, we discover a curl-free vector field which is not conservative. 

(a) Define the vector field $\vec{F}(x,y,z) = \displaystyle{\left(\frac{-y}{x^2+y^2}, \frac{x}{x^2+y^2}, 0\right)}$. Show that $\nabla \times \vec{F}=\vec{0}$. 

\iftoggle{solutions}{}{\vfill}

(b) Show that the line integral of $\vec{F}$ around the origin-centered unit circle in the $x$-$y$ plane does not vanish. 

\iftoggle{solutions}{}{\vfill}

(c) How do you reconcile parts (a) and (b)? 

\iftoggle{solutions}{}{\vfill}

\sol{(a) We calculate $\nabla \times \vec{F}=\displaystyle{\left(\pd{F_2}{x}-\pd{F_1}{y}\right)\vec{k}=\frac{x^2+y^2-x(2x)+(x^2+y^2)-y(2y)}{(x^2+y^2)^2}}\vec{k}=\vec{0}$. 

(b) The integral of $\vec{F}$ around the unit circle is 
\[
\int_C \vec{F}\cdot d\vec{s} = \int_{0}^{2\pi} (-\sin\theta,\cos\theta)\cdot (-\sin\theta,\cos\theta)\,d\theta =\int_0^{2\pi} 1\,d\theta = 2\pi \neq 0.
\]

(c) The vector field $\vec{F}$ is curl-free but not conservative. This does not contradict the fact that 
\[
\text{a vector field }\vec{F}:\R^3 \to \R^3 \text{ is curl-free if and only if it's conservative},
\]
because $\vec{F}$ is defined on $D=\R^3\setminus \{z\text{-axis}\}$, not on all of $\R^3$. In fact, the existence of curl-free nonconservative vector fields on $D$ requires that $D$ not be \textit{simply connected}, which means that there exist loops in $D$ which cannot be contracted to a point within $D$. In the present case, any loop which surrounds the $z$-axis has this property. 

This observation serves as a gateway to a very important theory in differential geometry called \textit{de Rham cohomology}. One generalizes vector fields to \textit{differential forms} and the properties \textit{curl-free} and \textit{conservative} to \textit{closed} and \textit{exact} and uses information about closed nonexact forms on $D$ to figure out things about the ``holes'' in the domain $D$. This is helpful because $D$ might live in some high dimensional Euclidean space $\R^n$ for which visualizing the shape of $D$ is difficult. 
}{}
\end{document}

