\documentclass[11pt]{article}

\usepackage[utf8x]{inputenc}
\usepackage[L7x]{fontenc}  
\usepackage[pdftex]{graphicx}
\usepackage{fourier}
\usepackage{amssymb}
\usepackage{amsmath}
\usepackage{wrapfig}
\usepackage{sectsty}
\usepackage{asymptote}
\usepackage{colonequals}
\usepackage{color}
\usepackage{calc}
\usepackage{etoolbox}

% COMMENT OUT THE SECOND LINE TO MAKE A HANDOUT WITHOUT SOLUTIONS
\newtoggle{solutions}
\toggletrue{solutions}

\usepackage{amsthm}
\theoremstyle{definition}
\newtheorem{theorem}{Theorem}
\newtheorem{defn}{Definition}
\newtheorem{lemma}{Lemma}
\newtheorem{corollary}{Corollary}
\newtheorem{exercise}{Exercise}

\sectionfont{\large}
\sectionfont{\normalsize}

\def\Arg{\mathop{\rm Arg}\nolimits}
\def\Res{\mathop{\rm Res}}
\renewcommand\Im{\mathop{\rm Im}\nolimits}
\renewcommand\Re{\mathop{\rm Re}\nolimits}
\newcommand\Arctan{\mathop{\rm Arctan}\nolimits}
\newcommand{\R}{\mathbb{R}}
\newcommand{\C}{\mathbb{C}}
\newcommand{\N}{\mathbb{N}}
\newcommand{\Z}{\mathbb{Z}}
\renewcommand{\P}{\mathbb{P}}
\newcommand{\Chat}{\hat{\mathbb{C}}}
\newcommand{\UHP}{\mathbb{H}}
\DeclareMathOperator{\area}{area}
\DeclareMathOperator{\dist}{dist}
\DeclareMathOperator{\interior}{int}
\DeclareMathOperator{\id}{id}

\pagestyle{empty}

\textwidth = 6.5 in
\textheight = 9 in
\oddsidemargin = -0.125 in
\evensidemargin = 0.0 in
\topmargin = -0.2 in
\headheight = 0.0 in
\headsep = 0.2 in
\parskip = 0.2 in
\parindent = 0.0 in

\def\inv{^{-1}}

\newcounter{prob}
	\setcounter{prob}{1}

\newcounter{subprob}
	\setcounter{subprob}{1}

\newcommand\itm{\theprob.  \stepcounter{prob}\setcounter{subprob}{1}}
\newcommand\subitm{(\alph{subprob}) \refstepcounter{subprob}}

\newcommand\sol[2]{\iftoggle{solutions}{\textit{Solution}. #1}{#2}}
%\newcommand\sol[2]{#2}


\newcommand{\problem}[1]{
\makebox[0.2cm]{\textbf{\itm}}  \begin{minipage}[t]{\linewidth-0.75cm}
#1
\end{minipage}
}

\newcommand\twomatrix[4]{
\left(
\begin{array}{cc}
#1 & #2 \\
#3 & #4
\end{array}
\right)
}

\renewcommand\vec[1]{\mathbf{#1}}

\begin{document}
\thispagestyle{empty}

\begin{center}
  18.022 Recitation Quiz \iftoggle{solutions}{(with solutions)}{} \\
  22 September 2014 
\end{center}

\itm Suppose that $A = \left( \begin{array}{cc} 4 & 0 \\ 0 & 3 \end{array}
\right)$ and $B = \left( \begin{array}{cc} 0 & -1 \\ 1 & 0 \end{array}
\right)$. We regard $A$ and $B$ as maps from $\R^2$ to $\R^2$ by matrix
multiplication (on the left, so $A$ evaluated at $(1,2)$ is $(4,6)$, for
example), and we denote by $C$ the unit circle centered at the origin.

(a) Describe the image of $C$ under the map $AB$.

\sol{The matrix $B$ rotates the plane 90 degrees
  counterclockwise. Therefore, the image of $C$ under $B$ is the unit
  circle rotated 90 degrees, which is equal to $C$. The matrix $A$
  stretches the plane by a factor of 4 in the $x$ direction and a factor of
  $3$ in the $y$ direction. Therefore, the image of $C$ under $AB$ is an
  ellipse centered at the origin with major axis of length 8 in the $x$
  direction and minor axis of length 6 in the $y$ direction.}{\vfill}

(b) Describe the image of $C$ under the map $BA$. 

\sol{If we apply $A$ first, then we get the ellipse described in the
  previous question. Rotating the ellipse 90 degrees counterclockwise gives
  the ellipse with major axis of length 8 in the $y$ direction and minor
  axis of length $6$ in the $x$ direction. }{\vfill}

\end{document}
