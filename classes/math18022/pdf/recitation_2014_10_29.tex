\documentclass[11pt]{article}

\usepackage[utf8x]{inputenc}
\usepackage[L7x]{fontenc}  
\usepackage[pdftex]{graphicx}
\usepackage{fourier}
\usepackage{hyperref}
\usepackage{amssymb}
\usepackage{amsmath}
\usepackage{wrapfig}
\usepackage{sectsty}
\usepackage{asymptote}
\usepackage{colonequals}
\usepackage{color}
\usepackage{calc}
\usepackage{etoolbox}

% COMMENT OUT THE SECOND LINE TO MAKE A HANDOUT WITHOUT SOLUTIONS
\newtoggle{solutions}
%\toggletrue{solutions}

\usepackage{amsthm}
\theoremstyle{definition}
\newtheorem{theorem}{Theorem}
\newtheorem{defn}{Definition}
\newtheorem{lemma}{Lemma}
\newtheorem{corollary}{Corollary}
\newtheorem{exercise}{Exercise}

\sectionfont{\large}
\sectionfont{\normalsize}

\def\Arg{\mathop{\rm Arg}\nolimits}
\def\Res{\mathop{\rm Res}}
\renewcommand\Im{\mathop{\rm Im}\nolimits}
\renewcommand\Re{\mathop{\rm Re}\nolimits}
\newcommand\Arctan{\mathop{\rm Arctan}\nolimits}
\newcommand{\R}{\mathbb{R}}
\newcommand{\C}{\mathbb{C}}
\newcommand{\N}{\mathbb{N}}
\newcommand{\Z}{\mathbb{Z}}
\renewcommand{\P}{\mathbb{P}}
\newcommand{\Chat}{\hat{\mathbb{C}}}
\newcommand{\UHP}{\mathbb{H}}
\DeclareMathOperator{\area}{area}
\DeclareMathOperator{\dist}{dist}
\DeclareMathOperator{\interior}{int}
\DeclareMathOperator{\id}{id}

\pagestyle{empty}

\textwidth = 6.5 in
\textheight = 9 in
\oddsidemargin = -0.125 in
\evensidemargin = 0.0 in
\topmargin = -0.2 in
\headheight = 0.0 in
\headsep = 0.2 in
\parskip = 0.2 in
\parindent = 0.0 in

\def\inv{^{-1}}

\newcounter{prob}
	\setcounter{prob}{1}

\newcounter{subprob}
	\setcounter{subprob}{1}

\newcommand\itm{\theprob.  \stepcounter{prob}\setcounter{subprob}{1}}
\newcommand\subitm{(\alph{subprob}) \refstepcounter{subprob}}

\newcommand\sol[2]{\iftoggle{solutions}{\textit{Solution}. #1}{#2}}
%\newcommand\sol[2]{#2}


\newcommand{\problem}[1]{
\makebox[0.2cm]{\textbf{\itm}}  \begin{minipage}[t]{\linewidth-0.75cm}
#1
\end{minipage}
}

\newcommand\twomatrix[4]{
\left(
\begin{array}{cc}
#1 & #2 \\
#3 & #4
\end{array}
\right)
}

\renewcommand\vec[1]{\mathbf{#1}}
\newcommand\pd[2]{\frac{\partial #1}{\partial #2}}

\begin{document}
\thispagestyle{empty}

\begin{center}
  18.022 Recitation Handout \iftoggle{solutions}{(with solutions)}{} \\
  29 October 2014 
\end{center}


\itm Minimize the function $f(x,y) = (x-y)^2$ subject to the constraint $xy = 1$ without using Lagrange multipliers. Verify that the method of Lagrange multipliers gives the same result.

\sol{Geometrically, we are asked to minimize the (squared) difference between $x$ and $y$ for a point on the graph of the function $y = \frac{1}{x}$. In fact, it is possible for $x-y$ to be zero, namely if $x=\pm 1$. Therefore, the minimum is zero, achieved at $(1,1)$ and $(-1,-1)$. 

The method of Lagrange multipliers tells us that extrema occur at solutions to the system $\nabla f = \lambda \nabla g$, where $g(x,y) = xy$. Differentiating, we obtain
\[
\left\{
\begin{array}{r@{\:}c@{\:}l}
 1 &=& xy \\
 2(x-y) &=& \lambda y \\
 -2(x-y) &=& \lambda x. 
\end{array} 
\right.
\]
Adding the second and third equations tells us that either $\lambda = 0$ or $x=y$. In the former case, the second equation implies $x = y$, so no matter what we have $x=y$. Substituting into the first equation gives $x = \pm 1$, as desired. 
}{\vfill}

%\iftoggle{solutions}{}{\vfill} 

\itm (4.3.19 in \emph{Colley}) Find the maximum and minimum values of $f(x,y) = x^2+xy+y^2$ on the closed disk $D=\{ (x,y): x^2+y^2 \leq 4 \}$. Can you do it without using Lagrange multipliers? (Hint: $(x\pm y)^2 \geq 0$.)

\sol{ We have $\nabla f = (2x+y,x+2y)$ and $\nabla g = (2x,2y)$. The system $\nabla f = \lambda \nabla g$ gives  
\[
\left\{
\begin{array}{r@{\:}c@{\:}l}
 2x+y &=& 2\lambda x \\
 x+2y &=& 2\lambda y. 
\end{array} 
\right.
\]
Solving this system, we get either $x=y=0$ or $2(\lambda - 1) = \pm 1$, i.e. $x=\pm y$. Suppose $x=y$ and $x^2+y^2 \leq 4$. Then $f(x,y)$ is maximized at $\pm (\sqrt{2},\sqrt{2})$ with $f(x,y) = 6$ and minimized at $(x,y)=(0,0)$ with $f(x,y)=0$. Now suppose $x=-y$ and $x^2+y^2 \leq 4$, $f(x,y)$ is maximized at $(-\sqrt{2},\sqrt{2})$ and $(\sqrt{2},-\sqrt{2})$ with $f(x,y) = 2$ and again minimized at $(x,y)=(0,0)$. Combining these two cases, we conclude that the maximum value and the minimum value of $f$ are 6 and 0, respectively.

Without using Lagrange multipliers: Expanding $(x-y)^2 \geq 0$, we obtain
$xy \leq \frac{1}{2} (x^2+y^2)$. Therefore
\[ f(x,y) = x^2+y^2 + xy \leq \frac{3}{2}(x^2+y^2) \leq 6. \]
This means the value of $f(x,y)$ inside the disk $D$ cannot exceed 6. On the other hand, we see that if $x=y=\sqrt{2}$ then $f(x,y)=6$. So the maximum value of $f$ is indeed 6. To find the minimum value of $f$, we start with $(x+y)^2\geq 0$, or $xy \geq -\frac{1}{2}(x^2+y^2)$. So
\[ f(x,y) = x^2+y^2 + xy \geq \frac{1}{2}(x^2+y^2) \geq 0. \]
Finally $f(0,0)= 0$ justifies the minimum value of $f$.
}{\vfill}


\iftoggle{solutions}{}{\newpage}

\itm (4.3.24 in \emph{Colley}) Heron's formula for the area of a triangle whose sides have lengths $x,y,$ and $z$ is
\[ \text{Area } = \sqrt{s(s-x)(s-y)(s-z)}, \] where $s =
\frac{1}{2}(x+y+z)$ is the so-called semi-perimeter of the triangle. Use
Heron's formula to show that, for a fixed perimeter $P$, the triangle with
the largest area is equilateral.

\sol{Equivalently, we want to maximize the function 
\[ f(x,y,z) = (P-2x)(P-2y)(P-2z) \]
subject to the constraint $g(x,y,z) = x+y+z = P$. Note that $\nabla g = (1,1,1)$ and
\[ \nabla f = -2((P-2y)(P-2z),(P-2x)(P-2z),(P-2x)(P-2y)). \]
So, by the method of Lagrange multipliers, the maxima occur only when 
\[ (P-2x)(P-2y) = (P-2y)(P-2z) = (P-2z)(P-2x). \]
Since none of $(P-2x),(P-2y),(P-2z)$ can be zero (why?), the only case is when $P-2x=P-2y=P-2z$, i.e. $x=y=z$. 

}{\vfill} 


\end{document}

