\documentclass[11pt]{article}

\usepackage[utf8x]{inputenc}
\usepackage[L7x]{fontenc}  
\usepackage[pdftex]{graphicx}
\usepackage{fourier}
\usepackage{amssymb}
\usepackage{amsmath}
\usepackage{wrapfig}
\usepackage{sectsty}
\usepackage{asymptote}
\usepackage{colonequals}
\usepackage{color}
\usepackage{calc}
\usepackage{etoolbox}

% COMMENT OUT THE SECOND LINE TO MAKE A HANDOUT WITHOUT SOLUTIONS
\newtoggle{solutions}
%\toggletrue{solutions}

\usepackage{amsthm}
\theoremstyle{definition}
\newtheorem{theorem}{Theorem}
\newtheorem{defn}{Definition}
\newtheorem{lemma}{Lemma}
\newtheorem{corollary}{Corollary}
\newtheorem{exercise}{Exercise}

\sectionfont{\large}
\sectionfont{\normalsize}

\def\Arg{\mathop{\rm Arg}\nolimits}
\def\Res{\mathop{\rm Res}}
\renewcommand\Im{\mathop{\rm Im}\nolimits}
\renewcommand\Re{\mathop{\rm Re}\nolimits}
\newcommand\Arctan{\mathop{\rm Arctan}\nolimits}
\newcommand{\R}{\mathbb{R}}
\newcommand{\C}{\mathbb{C}}
\newcommand{\N}{\mathbb{N}}
\newcommand{\Z}{\mathbb{Z}}
\renewcommand{\P}{\mathbb{P}}
\newcommand{\Chat}{\hat{\mathbb{C}}}
\newcommand{\UHP}{\mathbb{H}}
\DeclareMathOperator{\area}{area}
\DeclareMathOperator{\dist}{dist}
\DeclareMathOperator{\interior}{int}
\DeclareMathOperator{\id}{id}

\pagestyle{empty}

\textwidth = 6.5 in
\textheight = 9 in
\oddsidemargin = -0.125 in
\evensidemargin = 0.0 in
\topmargin = -0.2 in
\headheight = 0.0 in
\headsep = 0.2 in
\parskip = 0.2 in
\parindent = 0.0 in

\def\inv{^{-1}}

\newcounter{prob}
	\setcounter{prob}{1}

\newcounter{subprob}
	\setcounter{subprob}{1}

\newcommand\itm{\theprob.  \stepcounter{prob}\setcounter{subprob}{1}}
\newcommand\subitm{(\alph{subprob}) \refstepcounter{subprob}}

\newcommand\sol[2]{\iftoggle{solutions}{\textit{Solution}. #1}{#2}}
%\newcommand\sol[2]{#2}


\newcommand{\problem}[1]{
\makebox[0.2cm]{\textbf{\itm}}  \begin{minipage}[t]{\linewidth-0.75cm}
#1
\end{minipage}
}

\newcommand\twomatrix[4]{
\left(
\begin{array}{cc}
#1 & #2 \\
#3 & #4
\end{array}
\right)
}

\renewcommand\vec[1]{\mathbf{#1}}

\begin{document}
\thispagestyle{empty}

\begin{center}
  18.022 Recitation Quiz \iftoggle{solutions}{(with solutions)}{} \\
  3 November 2014 
\end{center}


\itm Evaluate $\displaystyle{\int_{0}^{\pi}\!\int_{0}^{x} \cos(x+y) \,dy\,dx}$ and sketch the region of integration in $\R^2$ indicated by the limits of integration. 

\sol{
The domain of integration is shown below. 
\begin{center}
  \includegraphics{figures/region2}
\end{center}
Evaluating the integral, we find 
  \begin{align*}
    \int_{0}^{\pi}\int_{0}^{y}\cos(x+y) \,dy\,dx &= \int_{0}^{\pi} \left[\sin(x+y)\right]_{0}^{x}\,dx  \\
    &= \int_0^\pi \sin(2x)-\sin(x) \,dx \\
    &= \boxed{-2}. \qedhere
  \end{align*}
}{\vfill}

\itm Let $f:\R^2 \to \R$ be a continuous function, and consider the integral $\displaystyle{\int_{0}^{1}\!\int_{-x^2}^{x^2}f(x,y)\,dy\,dx}$. 

(a) Sketch the region of integration. 

\iftoggle{solutions}{}{\vfill} 

(b) Rewrite the integral with the order of integration switched. 

\sol{
  (a) The region of integration is shown below. 
  \begin{center}
    \includegraphics{figures/region3} 
  \end{center} 
  (b) Looking at the region in part (a) to change the order of integration, we get
  \[
  \int_{0}^{1}\int_{-x^2}^{x^2}f(x,y)\,dy\,dx = \int_{-1}^{1}\int_{\sqrt{|y|}}^{1}f(x,y)\,dx\,dy. \qedhere
  \]
}{\vfill}


\end{document}
