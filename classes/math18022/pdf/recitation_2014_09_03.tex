\documentclass[11pt]{article}

\usepackage[utf8x]{inputenc}
\usepackage[L7x]{fontenc}  
\usepackage[pdftex]{graphicx}
\usepackage{fourier}
\usepackage{amssymb}
\usepackage{amsmath}
\usepackage{wrapfig}
\usepackage{sectsty}
\usepackage{asymptote}
\usepackage{colonequals}
\usepackage{color}
\usepackage{calc}
\usepackage{etoolbox}

% COMMENT OUT THE SECOND LINE TO MAKE A HANDOUT WITHOUT SOLUTIONS
\newtoggle{solutions}
\toggletrue{solutions}


\usepackage{amsthm}
\theoremstyle{definition}
\newtheorem{theorem}{Theorem}
\newtheorem{defn}{Definition}
\newtheorem{lemma}{Lemma}
\newtheorem{corollary}{Corollary}
\newtheorem{exercise}{Exercise}

\sectionfont{\large}
\sectionfont{\normalsize}

\def\Arg{\mathop{\rm Arg}\nolimits}
\def\Res{\mathop{\rm Res}}
\renewcommand\Im{\mathop{\rm Im}\nolimits}
\renewcommand\Re{\mathop{\rm Re}\nolimits}
\newcommand\Arctan{\mathop{\rm Arctan}\nolimits}
\newcommand{\R}{\mathbb{R}}
\newcommand{\C}{\mathbb{C}}
\newcommand{\N}{\mathbb{N}}
\newcommand{\Z}{\mathbb{Z}}
\renewcommand{\P}{\mathbb{P}}
\newcommand{\Chat}{\hat{\mathbb{C}}}
\newcommand{\UHP}{\mathbb{H}}
\DeclareMathOperator{\area}{area}
\DeclareMathOperator{\dist}{dist}
\DeclareMathOperator{\interior}{int}
\DeclareMathOperator{\id}{id}

\pagestyle{empty}

\textwidth = 6.5 in
\textheight = 9 in
\oddsidemargin = -0.025 in
\evensidemargin = 0.0 in
\topmargin = -0.2 in
\headheight = 0.0 in
\headsep = 0.2 in
\parskip = 0.2 in
\parindent = 0.0 in

\def\inv{^{-1}}

\newcounter{prob}
	\setcounter{prob}{1}

\newcommand\itm{\theprob.  \stepcounter{prob}}

\newcommand\sol[2]{\iftoggle{solutions}{\begin{proof}[Solution] #1\end{proof}}{#2}}

%\newcommand\sol{\textit{Solution}. }

\newcommand{\problem}[1]{
\makebox[0.2cm]{\textbf{\itm}}  \begin{minipage}[t]{\linewidth-0.75cm}
#1
\end{minipage}
}

\begin{document}
\thispagestyle{empty}

\begin{center}
  18.022 Recitation Handout \\
  3 September 2014 \\
\end{center}

\itm Leona discovered a set of instructions for finding a buried treasure. The instructions say:
\begin{enumerate}
\item Start at the old oak tree in the downtown park. 
\item Walk 14 meters northeast. 
\item Walk 12 meters west. 
\item Walk 10 meters north. 
\item Walk 14 meters southeast. 
\item Walk 8 meters east. 
\item Dig a hole two feet deep. 
\end{enumerate}
She hops on a train to make her way downtown, and during the ride she decides to simplify the instructions as much as possible. Replace steps 2-6 with at most two instructions which will still get Leona to the buried treasure. 


\itm A rope of length $12\pi$ units is partially wrapped around a tree of radius 12 units, as shown in the figure below. The part of the rope not touching the tree is pulled tight. Find the coordinates of the end of the rope, labeled $E$. 
\begin{center}
  \includegraphics{figures/ropetree}
\end{center}

\newpage

\itm Repeat the previous exercise but with a general angle $\theta$ in place of $60^\circ$, a general radius $r$ in place of 12, and a general rope length $L$ in place of $12\pi$. 
\begin{center}
  \includegraphics{figures/ropetree2}
\end{center}

\itm (Fun/Challenge Problem) You are at the center of a circular pond, and
at the perimeter there is a monster who cannot swim. The monster can run 4
times faster than your swimming speed, and both of you can change direction
instantaneously at will. You can run faster than the monster, so if you
reach the edge at a point where the monster is not waiting to attack, you
can escape. (a) Describe a strategy for escaping the monster that works no
matter how it chooses to move. (b) Find all numbers with which 4 may be
replaced such that it is still possible to escape the monster. 

\begin{center}
  \includegraphics{figures/monster}
\end{center}

\end{document}
