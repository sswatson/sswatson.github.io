\documentclass[11pt]{article}

\usepackage[utf8x]{inputenc}
\usepackage[T1]{fontenc}  
\usepackage[pdftex]{graphicx}
\usepackage{hyperref}
\usepackage{amsmath}
\usepackage{amssymb}
\usepackage{wrapfig}
\usepackage{sectsty}
\usepackage{colonequals}
\usepackage{color}
\usepackage{calc}
\usepackage{etoolbox}

% COMMENT OUT THE SECOND LINE TO MAKE A HANDOUT WITHOUT SOLUTIONS
\newtoggle{solutions}
%\toggletrue{solutions}

\usepackage{amsthm}
\theoremstyle{definition}
\newtheorem{theorem}{Theorem}
\newtheorem{defn}{Definition}
\newtheorem{lemma}{Lemma}
\newtheorem{corollary}{Corollary}
\newtheorem{exercise}{Exercise}

\usepackage{palatino} 
\usepackage{pxfonts} 

\sectionfont{\large}
\sectionfont{\normalsize}

\def\Arg{\mathop{\rm Arg}\nolimits}
\def\Res{\mathop{\rm Res}}
\renewcommand\Im{\mathop{\rm Im}\nolimits}
\renewcommand\Re{\mathop{\rm Re}\nolimits}
\newcommand\Arctan{\mathop{\rm Arctan}\nolimits}
\newcommand{\R}{\mathbb{R}}
\newcommand{\C}{\mathbb{C}}
\newcommand{\N}{\mathbb{N}}
\newcommand{\Z}{\mathbb{Z}}
\renewcommand{\P}{\mathbb{P}}
\newcommand{\Chat}{\hat{\mathbb{C}}}
\newcommand{\UHP}{\mathbb{H}}
\DeclareMathOperator{\area}{area}
\DeclareMathOperator{\dist}{dist}
\DeclareMathOperator{\interior}{int}
\DeclareMathOperator{\id}{id}

\pagestyle{empty}

\textwidth = 6.5 in
\textheight = 9 in
\oddsidemargin = -0.125 in
\evensidemargin = 0.0 in
\topmargin = -0.2 in
\headheight = 0.0 in
\headsep = 0.2 in
\parskip = 0.2 in
\parindent = 0.0 in

\def\inv{^{-1}}

\newcounter{prob}
	\setcounter{prob}{1}

\newcounter{subprob}
	\setcounter{subprob}{1}

\newcommand\itm{\theprob.  \stepcounter{prob}\setcounter{subprob}{1}}
\newcommand\subitm{(\alph{subprob}) \refstepcounter{subprob}}

\newcommand\sol[2]{\iftoggle{solutions}{\begin{proof}[Solution] #1\end{proof}}{#2}}
%\newcommand\sol[2]{\iftoggle{solutions}{\textit{Solution}. #1}{#2}}
%\newcommand\sol[2]{#2}


\newcommand{\problem}[1]{
\makebox[0.2cm]{\textbf{\itm}}  \begin{minipage}[t]{\linewidth-0.75cm}
#1
\end{minipage}
}

\newcommand\twomatrix[4]{
\left(
\begin{array}{cc}
#1 & #2 \\
#3 & #4
\end{array}
\right)
}

\renewcommand\vec[1]{\mathbf{#1}}
\newcommand\pd[2]{\frac{\partial #1}{\partial #2}}

\begin{document}
\thispagestyle{empty}

\begin{center}
  18.022 Recitation Handout \iftoggle{solutions}{(with solutions)}{} \\
  24 November 2014 \\
\end{center}

\itm According to Coulomb's law, the force between a particle of charge $q_1$ at the origin and a particle of charge $q_2$ at the point $\vec{r} = (x,y,z)\in \R^3$ is given by 
\[
\mathbf{F} = \frac{q_1q_2}{4\pi \varepsilon_0}\frac{\vec{r}}{|\vec{r}|^3},
\]
where $\varepsilon_0$ is a physical constant. 

(a) Is $\vec{F}$ a conservative vector field? If so, find a function $\phi:\R^3 \to \R$ such that $\nabla \phi = \vec{F}$. 

\iftoggle{solutions}{}{\vfill}

(b) If the distance between two charges is tripled, by what factor is the force between them reduced? 

\iftoggle{solutions}{}{\vfill}

(c) How much work is required to move the second particle along the path 
\[
\vec{\gamma}(t)=(1+(1-t)\cos(t^2),\sqrt{\sin{\pi t}},4t - t^2) \qquad 0\leq t \leq 1?
\] 
Express your answer in terms of $q_1$, $q_2$, and $\varepsilon_0$. 

\iftoggle{solutions}{}{\vfill}

\sol{(a) Writing $\vec{F}$ as 
\[
\frac{q_1q_2}{4\pi \varepsilon_0}\left(\frac{x}{(x^2+y^2+z^2)^{3/2}},\frac{y}{(x^2+y^2+z^2)^{3/2}},\frac{z}{(x^2+y^2+z^2)^{3/2}}\right),
\]
we see that $\vec{F}$ is the gradient of 
\[
\phi(x,y,z) = -\frac{q_1q_2}{4\pi \varepsilon_0}\frac{1}{\sqrt{x^2+y^2+z^2}} = \boxed{-\frac{q_1q_2}{4\pi \varepsilon_0}\frac{1}{|\vec{r}|}}.
\]
Therefore, $\vec{F}$ is conservative. 

(b) The magnitude of $\vec{F}$ is proportional to $|\vec{r}|/|\vec{r}|^3=|\vec{r}|^{-2},$ so tripling the distance decreases the force by a factor of $\boxed{9}$.

(c) The amount of work required to move the particle from a point $\vec{r}_1$ to a point $\vec{r}_2$ along a path $\gamma$ is $\int_\gamma \vec{F}\cdot d\vec{s}$. Since $\vec{F}=\nabla \phi$ is conservative, the value of this integral is $\phi(\vec{r}_2) - \phi(\vec{r}_1)$ no matter what path $\gamma$ from $\vec{r}_1$ to $\vec{r}_2$ is chosen. For the given path, the starting point is $\gamma(0) = (2,0,0)$ and the ending point is $(1,0,3)$. Thus the work is $\boxed{\frac{q_1q_2}{4\pi \varepsilon_0} \left(\frac{1}{2}-\frac{1}{\sqrt{10}}\right)}$.
}{}

\newpage 

\itm (6.2.23 in \textit{Colley}) Let $D$ be a region to which Green's theorem applies and suppose that $u(x,y)$ and $v(x,y)$ are two functions of class $C^2$ whose domains include $D$. Show that 
\[
\iint_D \frac{\partial (u,v)}{\partial(x,y)}\,dA = \oint_C (u\nabla v)\cdot d\vec{s},
\]
where $C=\partial D$ is oriented as in Green's theorem. 

\sol{Writing out the right-hand side and applying Green's theorem, we get 
  \begin{align*}
    \oint_C (u\nabla v)\cdot d\vec{s} &= \int uv_x\,dx + uv_y\,dy \\
    &= \iint_D \frac{\partial}{\partial x}(uv_y)- \frac{\partial}{\partial y}(uv_x)\, dA\\
    &= \iint_D u_x v_y - u_yv_x \,dA\\
    &= \iint_D \frac{\partial (u,v)}{\partial(x,y)}\,dA. \qedhere
  \end{align*}
}{\vfill}

\itm (6.1.29 in \textit{Colley}) Let $C$ be a level set of the function $f(x,y)$. Show that $\int_C \nabla f\cdot d\vec{s} = 0$. 

\sol{
  Letting $a$ and $b$ be the endpoints of the curve $C$, we calculate $\int_C \nabla f\cdot d\vec{s} = f(b) - f(a) = 0$, since $f$ is constant on $C$. 
}{\vfill}

\end{document}

