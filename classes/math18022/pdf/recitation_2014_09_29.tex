\documentclass[11pt]{article}

\usepackage[utf8x]{inputenc}
\usepackage[L7x]{fontenc}  
\usepackage[pdftex]{graphicx}
\usepackage{fourier}
\usepackage{amssymb}
\usepackage{amsmath}
\usepackage{wrapfig}
\usepackage{sectsty}
\usepackage{asymptote}
\usepackage{colonequals}
\usepackage{color}
\usepackage{calc}
\usepackage{etoolbox}

% COMMENT OUT THE SECOND LINE TO MAKE A HANDOUT WITHOUT SOLUTIONS
\newtoggle{solutions}
%\toggletrue{solutions}

\usepackage{amsthm}
\theoremstyle{definition}
\newtheorem{theorem}{Theorem}
\newtheorem{defn}{Definition}
\newtheorem{lemma}{Lemma}
\newtheorem{corollary}{Corollary}
\newtheorem{exercise}{Exercise}

\sectionfont{\large}
\sectionfont{\normalsize}

\def\Arg{\mathop{\rm Arg}\nolimits}
\def\Res{\mathop{\rm Res}}
\renewcommand\Im{\mathop{\rm Im}\nolimits}
\renewcommand\Re{\mathop{\rm Re}\nolimits}
\newcommand\Arctan{\mathop{\rm Arctan}\nolimits}
\newcommand{\R}{\mathbb{R}}
\newcommand{\C}{\mathbb{C}}
\newcommand{\N}{\mathbb{N}}
\newcommand{\Z}{\mathbb{Z}}
\renewcommand{\P}{\mathbb{P}}
\newcommand{\Chat}{\hat{\mathbb{C}}}
\newcommand{\UHP}{\mathbb{H}}
\DeclareMathOperator{\area}{area}
\DeclareMathOperator{\dist}{dist}
\DeclareMathOperator{\interior}{int}
\DeclareMathOperator{\id}{id}

\pagestyle{empty}

\textwidth = 6.5 in
\textheight = 9 in
\oddsidemargin = -0.125 in
\evensidemargin = 0.0 in
\topmargin = -0.2 in
\headheight = 0.0 in
\headsep = 0.2 in
\parskip = 0.2 in
\parindent = 0.0 in

\def\inv{^{-1}}

\newcounter{prob}
	\setcounter{prob}{1}

\newcounter{subprob}
	\setcounter{subprob}{1}

\newcommand\itm{\theprob.  \stepcounter{prob}\setcounter{subprob}{1}}
\newcommand\subitm{(\alph{subprob}) \refstepcounter{subprob}}

\newcommand\sol[2]{\iftoggle{solutions}{\textit{Solution}. #1}{#2}}
%\newcommand\sol[2]{#2}


\newcommand{\problem}[1]{
\makebox[0.2cm]{\textbf{\itm}}  \begin{minipage}[t]{\linewidth-0.75cm}
#1
\end{minipage}
}

\newcommand\twomatrix[4]{
\left(
\begin{array}{cc}
#1 & #2 \\
#3 & #4
\end{array}
\right)
}

\renewcommand\vec[1]{\mathbf{#1}}
\newcommand\pd[2]{\frac{\partial #1}{\partial #2}}

\begin{document}
\thispagestyle{empty}

\begin{center}
  18.022 Recitation Handout \iftoggle{solutions}{(with solutions)}{} \\
  29 September 2014
\end{center}

\itm Evaluate the limit $\lim_{(x,y,z)\to(0,0,0)}\frac{2xy^2+xy}{x^2+y^2}$, or explain why the limit fails to exist. Recall that the limit exists and equals $L$ if for all $\epsilon>0$ there exists $\delta>0$ which may depend on $\epsilon$ such that whenever $0<\|(x,y,z)\|<\delta$, the value of $\frac{2xy^2+xy}{x^2+y^2}$ is within $\epsilon$ of $L$. If the limit exists, find some $\delta$ that satisfies this definition. If the limit does not exist, find a value of $\epsilon$ for which no value of $\delta$ and $L$ satisfy the definition.

\sol{The limit does not exist, because we can achieve different limits along different paths going to $(0,0,0)$. For example, if we take the limit along the $x$-axis, we get $\lim_{(x,0,0)\to(0,0,0)}\frac{0}{x^2}=0$. If we take the limit along the line $y=x$, we get $\lim_{(x,x,0)\to(0,0,0)}\frac{2x^3+x^2}{2x^2}=1/2$. Any value of $\epsilon$ less than 1/2 will work to show that no $\delta$ and $L$ could satisfy the limit. }{\vfill}

\itm Find $\pd{f}{x}$, $\pd{f}{y}$, and $\pd{f}{z}$ for $f(x,y,z) = \frac{1}{xy} + \log(xyz) + e^x\sin(y z) $. 

\sol{$\pd{f}{x}=\boxed{-\frac{1}{x^2y} + \frac{1}{x} + e^x \cos(yz)}$,
  $\pd{f}{y}=\boxed{-\frac{1}{y^2x} + \frac{1}{y} + ze^x \sin(yz)}$, and
  $\pd{f}{z}=\boxed{\frac{1}{z} + ye^x \cos(yz)}$}{\vfill}

\iftoggle{solutions}{}{\newpage}

\itm Define $f(x) = (x^2 - 4)/(x-2)$ when $x\neq 2$ and $f(2) = c_1$. Determine the value of the constant $c_1$ for which $f$ is continuous. Do the same for 
\[
g(x,y) = 
\left\{
\begin{array}{cl}
\frac{3|x|^3+3|y|^3-x^{10}\arctan(y)}{|x|^3+|y|^3} & \text{ if}(x,y)\neq (0,0) \\
c_2 & \text{ if}(x,y)=(0,0).
\end{array} 
\right.
\]

\sol{$f(x) = x+2$ when $x\neq 2$, so the limit as $x\to 2$ of $f(x)=4$. Since a function is continuous at 2 if the limit at 2 exists and equals the value of the function $c_1=4$ is the right choice. 

When $(x,y) \neq (0,0)$, $g(x,y)$ equals $3-\frac{x^{10}}{|x|^3+|y|^3}\arctan(y)$. As $(x,y)\to (0,0)$, the second term goes to 0. One way to see this is to use the boundedness of arctan and switch the rest of the expression to polar coordinates. So the limit of $g(x,y)$ as $(x,y)\to (0,0)$ is $\boxed{3}$. }{\vfill}


\itm Suppose that $f,g,$ and $h$ are real-valued functions on $\R^n$, and suppose that $f(\vec{x}) = g(\vec{x})h(\vec{x})$. Prove that if $g$ is bounded and $\lim_{\vec{x}\to \vec{0}}h(\vec{x}) = 0$, then $\lim_{\vec{x}\to \vec{0}}f(\vec{x}) = 0$.

\sol{Suppose that $B$ is a positive bound for $|g|$, so that $-B \leq g(\vec{x})\leq B$ for all $\vec{x}$. By the definition of a limit, for every $e>0$, there exists $\delta>0$ so that $|h(\vec{x})|<e$ for all $\vec{x}$ satisfying $0<|\vec{x}|<\delta$. 

So given $\epsilon>0$, we can feed $e=\epsilon/B$ into the preceding sentence to find $\delta$ so that $|h(\vec{x})|<\epsilon/B$ whenever $0<|\vec{x}|<\delta$. This value of $\delta$ has the property that whenever $0<|\vec{x}|<\delta$, we have $|f(\vec{x})|=|g(\vec{x})h(\vec{x})|\leq B(\epsilon/B) = \epsilon$. In other words, this value of $\delta$ satisfies the definition of $\lim_{\vec{x}\to 0} f(\vec{x}) = 0.$}{\vfill} 

\end{document}

