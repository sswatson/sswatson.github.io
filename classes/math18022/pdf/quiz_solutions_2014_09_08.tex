\documentclass[11pt]{article}

\usepackage[utf8x]{inputenc}
\usepackage[L7x]{fontenc}  
\usepackage[pdftex]{graphicx}
\usepackage{fourier}
\usepackage{amssymb}
\usepackage{amsmath}
\usepackage{wrapfig}
\usepackage{sectsty}
\usepackage{asymptote}
\usepackage{colonequals}
\usepackage{color}
\usepackage{calc}

\usepackage{etoolbox}

% COMMENT OUT THE SECOND LINE TO MAKE A HANDOUT WITHOUT SOLUTIONS
\newtoggle{solutions}
\toggletrue{solutions}

\usepackage{amsthm}
\theoremstyle{definition}
\newtheorem{theorem}{Theorem}
\newtheorem{defn}{Definition}
\newtheorem{lemma}{Lemma}
\newtheorem{corollary}{Corollary}
\newtheorem{exercise}{Exercise}

\sectionfont{\large}
\sectionfont{\normalsize}

\def\Arg{\mathop{\rm Arg}\nolimits}
\def\Res{\mathop{\rm Res}}
\renewcommand\Im{\mathop{\rm Im}\nolimits}
\renewcommand\Re{\mathop{\rm Re}\nolimits}
\newcommand\Arctan{\mathop{\rm Arctan}\nolimits}
\newcommand{\R}{\mathbb{R}}
\newcommand{\C}{\mathbb{C}}
\newcommand{\N}{\mathbb{N}}
\newcommand{\Z}{\mathbb{Z}}
\renewcommand{\P}{\mathbb{P}}
\newcommand{\Chat}{\hat{\mathbb{C}}}
\newcommand{\UHP}{\mathbb{H}}
\DeclareMathOperator{\area}{area}
\DeclareMathOperator{\dist}{dist}
\DeclareMathOperator{\interior}{int}
\DeclareMathOperator{\id}{id}

\pagestyle{empty}

\textwidth = 6.5 in
\textheight = 9 in
\oddsidemargin = -0.125 in
\evensidemargin = 0.0 in
\topmargin = -0.2 in
\headheight = 0.0 in
\headsep = 0.2 in
\parskip = 0.2 in
\parindent = 0.0 in

\def\inv{^{-1}}

\newcounter{prob}
	\setcounter{prob}{1}

\newcommand\itm{\theprob.  \stepcounter{prob}}

\iftoggle{solutions}{
\newcommand\sol[2]{\textit{Solution}. #1}
}
{
\newcommand\sol[2]{#2}
}

\newcommand{\problem}[1]{
\makebox[0.2cm]{\textbf{\itm}}  \begin{minipage}[t]{\linewidth-0.75cm}
#1
\end{minipage}
}

\begin{document}
\thispagestyle{empty}

\begin{center}
  18.022 Recitation Quiz \iftoggle{solutions}{(with solutions)}{} \\
  8 September 2014 \\
\end{center}

\itm A circus ride consists of a chair revolving at 30 revolutions per
minute around a point which is itself revolving around different point at a
rate of 12 revolutions per minute, as shown in the figure. The radii of the
two circles are 2 meters and 8 meters, respectively. Assuming that the
chair starts 10 meters directly to the right of center of the larger
circle, write a parametric expression in terms of $t$ (measured in minutes)
which describes the position of the chair relative to the center of the
larger circle.

\begin{center} 
\includegraphics[width=4cm]{circus}
\end{center} 

\sol{The point on the larger circle is at the location $(\cos(2\pi
  t/30),\sin (2\pi t/30))$, while the vector from this point to the chair
  is $(\cos(2\pi t/12),\sin (2\pi t/12))$. The location of the chair is the
  vector sum of these vectors, which is 
\[
(\cos(2\pi t/30) + \cos(2\pi t/12),\sin (2\pi t/30)+\sin (2\pi t/12)).
\]
}{}

\end{document}
