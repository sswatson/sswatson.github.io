\documentclass[11pt]{article}

\usepackage[utf8x]{inputenc}
\usepackage[L7x]{fontenc}  
\usepackage[pdftex]{graphicx}
\usepackage{fourier}
\usepackage{amssymb}
\usepackage{amsmath}
\usepackage{wrapfig}
\usepackage{sectsty}
\usepackage{asymptote}
\usepackage{colonequals}
\usepackage{color}
\usepackage{calc}
\usepackage{etoolbox}

% COMMENT OUT THE SECOND LINE TO MAKE A HANDOUT WITHOUT SOLUTIONS
\newtoggle{solutions}
\toggletrue{solutions}

\usepackage{amsthm}
\theoremstyle{definition}
\newtheorem{theorem}{Theorem}
\newtheorem{defn}{Definition}
\newtheorem{lemma}{Lemma}
\newtheorem{corollary}{Corollary}
\newtheorem{exercise}{Exercise}

\sectionfont{\large}
\sectionfont{\normalsize}

\def\Arg{\mathop{\rm Arg}\nolimits}
\def\Res{\mathop{\rm Res}}
\renewcommand\Im{\mathop{\rm Im}\nolimits}
\renewcommand\Re{\mathop{\rm Re}\nolimits}
\newcommand\Arctan{\mathop{\rm Arctan}\nolimits}
\newcommand{\R}{\mathbb{R}}
\newcommand{\C}{\mathbb{C}}
\newcommand{\N}{\mathbb{N}}
\newcommand{\Z}{\mathbb{Z}}
\renewcommand{\P}{\mathbb{P}}
\newcommand{\Chat}{\hat{\mathbb{C}}}
\newcommand{\UHP}{\mathbb{H}}
\DeclareMathOperator{\area}{area}
\DeclareMathOperator{\dist}{dist}
\DeclareMathOperator{\interior}{int}
\DeclareMathOperator{\id}{id}

\pagestyle{empty}

\textwidth = 6.5 in
\textheight = 9 in
\oddsidemargin = -0.125 in
\evensidemargin = 0.0 in
\topmargin = -0.2 in
\headheight = 0.0 in
\headsep = 0.2 in
\parskip = 0.2 in
\parindent = 0.0 in

\def\inv{^{-1}}

\newcounter{prob}
	\setcounter{prob}{1}

\newcommand\itm{\theprob.  \stepcounter{prob}}

\newcommand\sol[2]{\iftoggle{solutions}{\textit{Solution}. #1}{#2}}
%\newcommand\sol[2]{#2}


\newcommand{\problem}[1]{
\makebox[0.2cm]{\textbf{\itm}}  \begin{minipage}[t]{\linewidth-0.75cm}
#1
\end{minipage}
}

\newcommand\twomatrix[4]{
\left(
\begin{array}{cc}
#1 & #2 \\
#3 & #4
\end{array}
\right)
}

\begin{document}
\thispagestyle{empty}

\begin{center}
  18.022 Recitation Handout \iftoggle{solutions}{(with solutions)}{} \\
  15 September 2014 
\end{center}

\itm Find the cosine of the angle between the vectors $(-4,2,6,5)$ and $(3,1,0,-7)$. 

\sol{The cosine of the angle between two vectors $\mathbf{a}$ and $\mathbf{b}$ is defined by $\cos\theta = \mathbf{a}\cdot \mathbf{b}/\left|\mathbf{a}\right|\left|\mathbf{b}\right|=(-45/(9\cdot \sqrt{59})=-5/\sqrt{59}.$}{\vfill}

\itm Is the cross-product associative? Is the dot product associative?
Prove or give a counterexample for each. 

\sol{The cross-product is not associative. For example, $(\mathbf{i}\times
  \mathbf{i})\times \mathbf{j}=0$ and $\mathbf{i}\times
  (\mathbf{i}\times \mathbf{j})=-\mathbf{j}$. The dot product is not
  associative because neither $(\mathbf{a}\cdot\mathbf{b})\cdot \mathbf{c}$
nor $\mathbf{a}\cdot(\mathbf{b}\cdot\mathbf{c})$ is defined. }{\vfill}

\itm Find an equation for the plane that contains the lines
\[
\left\{
\begin{array}{c@{\,}c@{\,}c}
  x(t)&=&5+t \\
  y(t)&=&1-t \\
  z(t)&=&4-3t. 
\end{array}
\right.
\]
and
\[
\left\{
\begin{array}{c@{\,}c@{\,}c}
  x(t)&=&5-4t \\
  y(t)&=&1+t \\
  z(t)&=&4-3t. 
\end{array}
\right.
\]

\sol{The plane contains the point $P_0=(5,1,4)$ and consists of all the
  points $P=(x,y,z)$ for which $\overrightarrow{P_0P}\cdot
  (\mathbf{v_1}\times\mathbf{v}_2)=0$, where $\mathbf{v}_1=(1,-1,-3)$ and
  $\mathbf{v}_2=(-4,1,-3)$. Since $\mathbf{v_1}\times\mathbf{v}_2 =
  (6,15,-3)$, the equation of the plane simplifies to $\boxed{6x + 15y - 3z =
  -33}$. }{\vfill}

\itm Find the distance from the origin to the plane $x+y+z=1$. 

\sol{The shortest line segment from the origin $O$ to the plane $x+y+z=1$
  intersects the plane at some point $P_0=(x_0,y_0,z_0)$. Since $\overrightarrow{PO}$ is
  perpendicular to the plane, the line containing $\overrightarrow{PO}$
  takes the form 
\[
\left\{
\begin{array}{c@{\,}c@{\,}c}
  x(t)&=&x_0 + t \\
  y(t)&=&y_0 + t \\
  z(t)&=&z_0 + t.
\end{array}
\right.
\]
Setting each of these coordinates equal to 0, we see that
$x_0=y_0=z_0$. Since the coordinates sum to 1, they are
$(1/3,1/3,1/3)$. Thus the distance is
$\sqrt{(1/3)^2+(1/3)^2+(1/3)^2}=\boxed{\sqrt{3}/3}$. 

We remark that it is also possible to solve this problem by computing the
volume of the tetrahedron with vertices at $O$, $(1,0,0)$, $(0,1,0)$, and
$(0,0,1$ in two different ways.  }{\vfill\newpage}


\itm Give a geometric description of what each of the following matrices does to a vector. 
\[
A=\twomatrix{1}{0}{0}{1} \qquad B=\twomatrix{3}{0}{0}{3} \qquad
C=\twomatrix{1}{0}{0}{-1} \qquad D = \twomatrix{0}{1}{1}{0} 
\]

\sol{$A$ does nothing, $B$ increases the length by a
    factor of 3, $C$ reflects across the $y$-axis, and $D$ reflects across
    the line $y=x$. }{\vfill}


\itm What matrices have the following geometric descriptions? 

(a) reflect across the $x$-axis

(b) reverse the direction of the vector and double its length

(c) halve the length of the vector (while preserving the direction)

(d) rotate a vector 90 degrees counterclockwise

(e) project a vector in $\mathbb{R}^3$ onto the $x$-$y$ plane

\sol{(a) $\twomatrix{-1}{0}{0}{1}$, \quad (b) $\twomatrix{-2}{0}{0}{-2}$, \quad (c)
  $\twomatrix{1/2}{0}{0}{1/2}$, \quad (d) $\twomatrix{0}{-1}{1}{0}$, \quad
(e) $\left(\begin{array}{ccc} 1 & 0 & 0 \\ 0 & 1 & 0 \end{array}\right)$, }

\iftoggle{solutions}{}{\vfill}

\itm The volume of a parallelepiped is the product of the area of its base
and its height. Consider the parallelepiped spanned by $\mathbf{a}$,
$\mathbf{b}$, and $\mathbf{c}$. You may suppose for simplicity that $\mathbf{b}$,
$\mathbf{c}$, and $\mathbf{a}$ form a right-handed triple of vectors, which
  means that a right-handed screw rotated an angle less than 180$^\circ$
  from $\mathbf{b}$ to $\mathbf{c}$ advances in the direction of
  $\mathbf{a}$. 

(a) Let us think of the parallelogram spanned by $\mathbf{b}$ and
$\mathbf{c}$ as the base of the parallelepiped. What is the (signed) area
of this parallelogram? 

(b) What is the height of the parallelogram, in terms of $\mathbf{a}$ and
the unit vector pointing in the direction of $\mathbf{b} \times\mathbf{c}$?

(c) Put together parts (a) and (b) to derive the triple scalar product
formula for the volume of a parallelepiped. 

\sol{The (signed) area of the parallelogram is
  $\left|\mathbf{b}\times\mathbf{c}\right|$. The height of the parallelogram is
  $\mathbf{a}\cdot\mathbf{u}$, where $\mathbf{u}$ is the unit vector in the
direction of $\mathbf{b}\times\mathbf{c}$. Therefore, the volume of the
parallelepiped is
\[
\mathbf{a}\cdot(\mathbf{u}\left|\mathbf{b}\times\mathbf{c}\right|)=\mathbf{a}\cdot(\mathbf{b}\times\mathbf{c}).
\]
}{\vfill\newpage}

\itm (Fun/Challenge problem) It is possible to prove the vector triple
product formula 
\[
\mathbf{a}\times(\mathbf{b}\times\mathbf{c}) =
(\mathbf{a}\cdot\mathbf{c})\mathbf{b} -
(\mathbf{a}\cdot\mathbf{b})\mathbf{c}
\]
in a tedious way using coordinates. This problem outlines a more conceptual
proof, taken from a note written by William C.\ Schulz. 

(a) Use the paralellepiped interpretation of the triple scalar product to show that 
\[
\mathbf{b}\cdot(\mathbf{c}\times\mathbf{n}) =
\mathbf{c}\cdot(\mathbf{n}\times\mathbf{b}) =
\mathbf{n}\cdot(\mathbf{b}\times\mathbf{c}) 
\]
(b) Use the right-hand rule to observe that if $\mathbf{c}$ is
perpendicular to $\mathbf{n}$, then
\[
\mathbf{n}\times(\mathbf{c}\times\mathbf{n}) = |\mathbf{n}|^2\mathbf{c}.
\]
(c) Show that it suffices to consider the case where $\mathbf{a}$,
$\mathbf{b}$, and $\mathbf{c}$ form a basis for $\mathbb{R}^3$.

(d) Write $\mathbf{a}\times(\mathbf{b}\times\mathbf{c})$ as a linear
combination of $\mathbf{a}, \mathbf{b},$ and $\mathbf{c},$ so that 
\begin{equation} \label{comp}
\mathbf{a}\times(\mathbf{b}\times\mathbf{c}) = \kappa \mathbf{a}+\lambda \mathbf{b}+\mu\mathbf{c},
\end{equation} 
(e) The easiest coefficient to determine is $\kappa$. What is it? 

(f) To find $\lambda$, dot both sides of \eqref{comp} with
$\mathbf{c}\times \mathbf{n},$ where $\mathbf{n}\colonequals
\mathbf{b}\times\mathbf{c}$. 

(g) To find $\mu$, dot both sides of \eqref{comp} with
$\mathbf{b}\times \mathbf{n},$ where $\mathbf{n}\colonequals
\mathbf{b}\times\mathbf{c}$. 

\sol{See http://mathdl.maa.org/images/cms\_upload/0746834213321.di020720.02p0099b.pdf.}{\vfill}

\end{document}
