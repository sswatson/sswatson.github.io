\documentclass[11pt]{article}

\usepackage[utf8x]{inputenc}
\usepackage[L7x]{fontenc}  
\usepackage[pdftex]{graphicx}
\usepackage{fourier}
\usepackage{hyperref}
\usepackage{amssymb}
\usepackage{amsmath}
\usepackage{wrapfig}
\usepackage{sectsty}
\usepackage{asymptote}
\usepackage{colonequals}
\usepackage{color}
\usepackage{calc}
\usepackage{etoolbox}

% COMMENT OUT THE SECOND LINE TO MAKE A HANDOUT WITHOUT SOLUTIONS
\newtoggle{solutions}
\toggletrue{solutions}

\usepackage{amsthm}
\theoremstyle{definition}
\newtheorem{theorem}{Theorem}
\newtheorem{defn}{Definition}
\newtheorem{lemma}{Lemma}
\newtheorem{corollary}{Corollary}
\newtheorem{exercise}{Exercise}

\sectionfont{\large}
\sectionfont{\normalsize}

\def\Arg{\mathop{\rm Arg}\nolimits}
\def\Res{\mathop{\rm Res}}
\renewcommand\Im{\mathop{\rm Im}\nolimits}
\renewcommand\Re{\mathop{\rm Re}\nolimits}
\newcommand\Arctan{\mathop{\rm Arctan}\nolimits}
\newcommand{\R}{\mathbb{R}}
\newcommand{\C}{\mathbb{C}}
\newcommand{\N}{\mathbb{N}}
\newcommand{\Z}{\mathbb{Z}}
\renewcommand{\P}{\mathbb{P}}
\newcommand{\Chat}{\hat{\mathbb{C}}}
\newcommand{\UHP}{\mathbb{H}}
\DeclareMathOperator{\area}{area}
\DeclareMathOperator{\dist}{dist}
\DeclareMathOperator{\interior}{int}
\DeclareMathOperator{\id}{id}

\pagestyle{empty}

\textwidth = 6.5 in
\textheight = 9 in
\oddsidemargin = -0.125 in
\evensidemargin = 0.0 in
\topmargin = -0.2 in
\headheight = 0.0 in
\headsep = 0.2 in
\parskip = 0.2 in
\parindent = 0.0 in

\def\inv{^{-1}}

\newcounter{prob}
	\setcounter{prob}{1}

\newcounter{subprob}
	\setcounter{subprob}{1}

\newcommand\itm{\theprob.  \stepcounter{prob}\setcounter{subprob}{1}}
\newcommand\subitm{(\alph{subprob}) \refstepcounter{subprob}}

\newcommand\sol[2]{\iftoggle{solutions}{\textit{Solution}. #1}{#2}}
%\newcommand\sol[2]{#2}


\newcommand{\problem}[1]{
\makebox[0.2cm]{\textbf{\itm}}  \begin{minipage}[t]{\linewidth-0.75cm}
#1
\end{minipage}
}

\newcommand\twomatrix[4]{
\left(
\begin{array}{cc}
#1 & #2 \\
#3 & #4
\end{array}
\right)
}

\renewcommand\vec[1]{\mathbf{#1}}
\newcommand\pd[2]{\frac{\partial #1}{\partial #2}}

\begin{document}
\thispagestyle{empty}

\begin{center}
  18.022 Recitation Handout \iftoggle{solutions}{(with solutions)}{} \\
  22 October 2014 
\end{center}


\itm Verify that divergence, curl, and gradient are linear operators. 

\sol{For divergence, we want to show that for all vector fields $\vec{F}$ and $\vec{G}$ and scalars $\alpha$ and $\beta$, we have 
\[
\nabla \cdot (\alpha \vec{F} + \beta \vec{G}) = \alpha \nabla \cdot \vec{F} + \beta \nabla \cdot \vec{G}. 
\]
The left-hand side is
 \[
\pd{}{x}(\alpha F_1 + \beta G_1) +\pd{}{y}(\alpha F_2 + \beta G_2) + \pd{}{z}(\alpha F_3 + \beta G_3)
 = 
\alpha \pd{F_1}{x}+ \beta \pd{G_1}{x} + \alpha \pd{F_2}{y}+ \beta \pd{G_2}{y} + \alpha \pd{F_3}{z}+ \beta \pd{G_3}{z},
\]
which equals the right-hand side. Calculations for gradient and curl are similar. 
}{\vfill}


\itm Let $\vec{F}(x,y,z)=(3x^2+\frac{1}{2}y^2+e^z, xy + z, f(x,y,z))$. Find all $f$ such that $\vec{F}$ is curl-free.

\sol{The third component of $\nabla \times \vec{F}$ is $y - (1/2)(2y) = 0$,
  as desired. For the first component to be zero, we must have $f_y = 1$,
  and for the second component to be zero we must have
  $f_x=e^z$. Integrating these two equations tells us that $f(x,y,z) = y +
  C_1(x,z)$ and $f(x,y,z) = xe^z + C_2(y,z)$ for functions $C_1$ and $C_2$
  which do not depend on $y$ or $x$ respectively. Putting these two
  together, we see that $f(x,y,z) = xe^z + y + C(z)$ for any differentiable
  function $C$. }{\vfill}

\iftoggle{solutions}{}{\newpage} 

\itm Confirm that for a vector field $\vec{F}:\R^3 \to \R^3$, we have 
\[
\nabla \times (\nabla \times \vec{F}) = \nabla(\nabla\cdot \vec{F}) - \nabla^2 \vec{F},
\]
where $\nabla^2 \vec{F}$ is defined to mean ``take the Laplacian of each component of $\vec{F}$.'' Is it possible to derive this identity from 
$
\vec{a}\times(\vec{b}\times\vec{c}) = (\vec{a}\cdot\vec{c})\vec{b} - (\vec{a}\cdot\vec{b})\vec{c}?
$
% $\left(\frac{\partial^2F_1}{\partial x^2}+\frac{\partial^2F_1}{\partial y^2}+\frac{\partial^2F_1}{\partial z^2}\right)\vec{i}$

\sol{[Omitted]}{\vfill}

\itm Let $\vec{F}$ be a $C^2$ vector field on $\R^3$. Show that $\nabla \times \vec{F}$ is incompressible. 

\sol{
We calculate 
\[
\nabla \cdot (\nabla \times \vec{F}) 
=
\frac{\partial^2 F_3}{\partial x \partial y} - \frac{\partial^2 F_2}{\partial x \partial z}
+ \frac{\partial^2 F_1}{\partial y \partial z} - \frac{\partial^2 F_3}{\partial y \partial x} 
+ \frac{\partial^2 F_2}{\partial z \partial x} - \frac{\partial^2 F_1}{\partial z \partial y} = 0,
\]
since the mixed partials don't depend on the order of differentiation, as $\vec{F}$ is $C^2$. 
}{\vfill}

\end{document}

