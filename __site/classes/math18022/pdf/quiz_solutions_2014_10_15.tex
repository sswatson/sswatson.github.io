\documentclass[11pt]{article}

\usepackage[utf8x]{inputenc}
\usepackage[L7x]{fontenc}  
\usepackage[pdftex]{graphicx}
%\usepackage{fourier}
\usepackage{amssymb}
\usepackage{amsmath}
\usepackage{wrapfig}
\usepackage{sectsty}
\usepackage{asymptote}
\usepackage{colonequals}
\usepackage{color}
\usepackage{calc}
\usepackage{etoolbox}

% COMMENT OUT THE SECOND LINE TO MAKE A HANDOUT WITHOUT SOLUTIONS
\newtoggle{solutions}
\toggletrue{solutions}

\usepackage{amsthm}
\theoremstyle{definition}
\newtheorem{theorem}{Theorem}
\newtheorem{defn}{Definition}
\newtheorem{lemma}{Lemma}
\newtheorem{corollary}{Corollary}
\newtheorem{exercise}{Exercise}

\sectionfont{\large}
\sectionfont{\normalsize}

\def\Arg{\mathop{\rm Arg}\nolimits}
\def\Res{\mathop{\rm Res}}
\renewcommand\Im{\mathop{\rm Im}\nolimits}
\renewcommand\Re{\mathop{\rm Re}\nolimits}
\newcommand\Arctan{\mathop{\rm Arctan}\nolimits}
\newcommand{\R}{\mathbb{R}}
\newcommand{\C}{\mathbb{C}}
\newcommand{\N}{\mathbb{N}}
\newcommand{\Z}{\mathbb{Z}}
\renewcommand{\P}{\mathbb{P}}
\newcommand{\Chat}{\hat{\mathbb{C}}}
\newcommand{\UHP}{\mathbb{H}}
\DeclareMathOperator{\area}{area}
\DeclareMathOperator{\dist}{dist}
\DeclareMathOperator{\interior}{int}
\DeclareMathOperator{\id}{id}

\pagestyle{empty}

\textwidth = 6.5 in
\textheight = 9 in
\oddsidemargin = -0.125 in
\evensidemargin = 0.0 in
\topmargin = -0.2 in
\headheight = 0.0 in
\headsep = 0.2 in
\parskip = 0.2 in
\parindent = 0.0 in

\def\inv{^{-1}}

\newcounter{prob}
	\setcounter{prob}{1}

\newcounter{subprob}
	\setcounter{subprob}{1}

\newcommand\itm{\theprob.  \stepcounter{prob}\setcounter{subprob}{1}}
\newcommand\subitm{(\alph{subprob}) \refstepcounter{subprob}}

\newcommand\sol[2]{\iftoggle{solutions}{\textit{Solution}. #1}{#2}}
%\newcommand\sol[2]{#2}


\newcommand{\problem}[1]{
\makebox[0.2cm]{\textbf{\itm}}  \begin{minipage}[t]{\linewidth-0.75cm}
#1
\end{minipage}
}

\newcommand\twomatrix[4]{
\left(
\begin{array}{cc}
#1 & #2 \\
#3 & #4
\end{array}
\right)
}

\renewcommand\vec[1]{\mathbf{#1}}

\begin{document}
\thispagestyle{empty}

\begin{center}
  18.022 Recitation Quiz \iftoggle{solutions}{(with solutions)}{} \\
  15 October 2014 
\end{center}

\itm Consider the function $f(x)=\sqrt{1-x^2}$ over the interval $[0,1]$. Write down a definite integral whose value is equal to the arclength of the graph of $f$.

\sol{
We calculate an arclength of
\[
\int_{0}^{1}\sqrt{1+f'(x)^2}\,dx = \int_{0}^{1}\sqrt{1+\left(\frac{-2x}{2\sqrt{1-x^2}}\right)^2}\,dx = \int_{0}^{1}\frac{1}{\sqrt{1-x^2}}\,dx.
\]
\textit{Remark}: Since this arclength is $\pi/2$ by the definition of
$\pi$, this exercise proves that
$\int_{0}^{1}\frac{1}{\sqrt{1-x^2}}\,dx=\pi/2$. Moreover, it can be
generalized to give a way of calculating the indefinite integral of
$1/\sqrt{1-x^2}$.  
} {\vfill}

\itm Consider the function $\mathbf{F}:\R^4\to\R^2$ defined by $\mathbf{F}(w,x,y,z)=(2/w^2-y,3x+\cos z)$.

(a) Find $D\mathbf{F}$.

\sol{The total derivative is the matrix of partial derivatives:
\[
\left(
  \begin{array}{cccc}
    -2/w^3 & 0 & -1 & 0 \\
    0 & 3 & 0 & -\sin z
  \end{array}
\right)
\]
}
{\vfill}

(b) Show that there exists an open set $U\subset \R^2$ containing $(1,2)$ and a function $\mathbf{f}:U\to \R^2$ such that for all $x\in U$, the equations 
$\mathbf{F}(w,x,y,z)=\mathbf{F}(1,2,3,\pi/2)$
have a unique solution $(y,z)=\mathbf{f}(w,x)$. Show that $\mathbf{f}$ is $C^1$. 

\sol{
The implicit function theorem ensures that we can solve (abstractly)
  for $(y,z)$ in terms of $x$ and $w$ if the matrix of partial derivatives
  corresponding to the $y$ and $z$ columns has nonvanishing determinant. In
  this case, that means 
\[
\det
\left(
  \begin{array}{cc}
    -1 & 0 \\
    0 & -\sin z
  \end{array}
\right)
= \det \left(
  \begin{array}{cc}
    -1 & 0 \\
    0 & -1
  \end{array}
\right)
= 1 \neq 0,
\]
so the implicit function theorem does apply. It ensures the existence of
such an $\vec{f}$ and the fact that $\vec{f}$ is $C^1$. 
}
{\vfill }

\vspace{0.5cm} 


\end{document}

