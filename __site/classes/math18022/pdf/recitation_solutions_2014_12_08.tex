\documentclass[11pt]{article}

\usepackage[utf8x]{inputenc}
\usepackage[T1]{fontenc}  
\usepackage[pdftex]{graphicx}
\usepackage{hyperref}
\usepackage{amsmath}
\usepackage{amssymb}
\usepackage{wrapfig}
\usepackage{sectsty}
\usepackage{colonequals}
\usepackage{color}
\usepackage{calc}
\usepackage{etoolbox}
\usepackage{multicol} 

% COMMENT OUT THE SECOND LINE TO MAKE A HANDOUT WITHOUT SOLUTIONS
\newtoggle{solutions}
\toggletrue{solutions}

\usepackage{amsthm}

\usepackage{palatino} 
\usepackage{pxfonts} 

\sectionfont{\large}
\sectionfont{\normalsize}

\def\Arg{\mathop{\rm Arg}\nolimits}
\def\Res{\mathop{\rm Res}}
\renewcommand\Im{\mathop{\rm Im}\nolimits}
\renewcommand\Re{\mathop{\rm Re}\nolimits}
\newcommand\Arctan{\mathop{\rm Arctan}\nolimits}
\newcommand{\R}{\mathbb{R}}
\newcommand{\C}{\mathbb{C}}
\newcommand{\N}{\mathbb{N}}
\newcommand{\Z}{\mathbb{Z}}
\renewcommand{\P}{\mathbb{P}}
\newcommand{\Chat}{\hat{\mathbb{C}}}
\newcommand{\UHP}{\mathbb{H}}
\DeclareMathOperator{\area}{area}
\DeclareMathOperator{\dist}{dist}
\DeclareMathOperator{\interior}{int}
\DeclareMathOperator{\id}{id}

\pagestyle{empty}

\textwidth = 6.5 in
\textheight = 9 in
\oddsidemargin = -0.125 in
\evensidemargin = 0.0 in
\topmargin = -0.2 in
\headheight = 0.0 in
\headsep = 0.2 in
\parskip = 0.2 in
\parindent = 0.0 in

\def\inv{^{-1}}

\newcounter{prob}
	\setcounter{prob}{1}

\newcounter{subprob}
	\setcounter{subprob}{1}

\newcommand\itm{\theprob.  \stepcounter{prob}\setcounter{subprob}{1}}
\newcommand\subitm{(\alph{subprob}) \refstepcounter{subprob}}

\newcommand\sol[2]{\iftoggle{solutions}{\begin{proof}[Solution] #1\end{proof}}{#2}}
%\newcommand\sol[2]{\iftoggle{solutions}{\textit{Solution}. #1}{#2}}
%\newcommand\sol[2]{#2}


\newcommand{\problem}[1]{
\makebox[0.2cm]{\textbf{\itm}}  \begin{minipage}[t]{\linewidth-0.75cm}
#1
\end{minipage}
}

\newcommand\twomatrix[4]{
\left(
\begin{array}{cc}
#1 & #2 \\
#3 & #4
\end{array}
\right)
}

\renewcommand\vec[1]{\mathbf{#1}}
\newcommand\pd[2]{\frac{\partial #1}{\partial #2}}


\newtheorem{theorem}{Theorem}
\newtheorem{lemma}[theorem]{Lemma}
\newtheorem{corollary}[theorem]{Corollary}
\newtheorem{example}[theorem]{Example}
\newtheorem{remark}[theorem]{Remark}
\newtheorem{bb}[theorem]{Blackboard}


\newcommand{\vecl}[1]{\overrightarrow{#1}}
\newcommand{\dls}[1]{\overrightarrow{#1}}

\def\vproj{\mbox{proj}}
\def\half{{\textstyle \frac12}}
\def\third{{\textstyle \frac13}}

\def\C{\mathbb{C}}
\def\R{\mathbb{R}}
\def\N{\mathbb{N}}
\def\vv{\vec{v}}
\def\vu{\vec{u}}
\def\vw{\vec{w}}
\def\vz{\vec{0}}
\def\va{\vec{a}}
\def\vb{\vec{b}}
\def\vc{\vec{c}}
\def\vr{\vec{r}}
\def\ih{\hat{\imath}}
\def\jh{\hat{\jmath}}
\def\kh{\hat{k}}
\def\eps{\epsilon}
\def\dd{\mathrm{d}}

\def\dir{\operatorname{div}}

\newcommand\blfootnote[1]{%
  \begingroup
  \renewcommand\thefootnote{}\footnote{#1}%
  \addtocounter{footnote}{-1}%
  \endgroup
}

\newcommand*\colvec[3][]{
    \begin{pmatrix}\ifx\relax#1\relax\else#1\\\fi#2\\#3\end{pmatrix}
}


\begin{document}
\thispagestyle{empty}

\begin{center}
  18.022 Recitation Handout \iftoggle{solutions}{(with solutions)}{} \\
  8 December 2014 \\
\end{center}

\newcommand{\itemt}[2]{\item {\bf {#1}} \newline{#2}}
%\newcommand{\solution}[1]{\vspace{0.15in}{\bf Solution:  {#1}}\vspace{0.15in}}
\newcommand{\solution}[1]{}



Let $D \subset \R^2$ be the region enclosed by the curve $r = g(\theta)$,
for some $\mathcal{C}^1$, non-negative $g \colon \R \to \R$ such that
$g(x+2\pi)=g(x)$ for all $x \in \R$.

\itm Calculate the length of $\partial D$, the boundary of
    $D$. Express your answer as an integral involving $g$ and its
    first derivative.

    \sol{ We parametrize the curve by $\vec r(\theta) =
      g(\theta)(\cos\theta, \sin\theta)$ for $\theta \in
      [0,2\pi]$. Then
      \begin{align*}
        \vec r'(\theta) = \frac{\partial
          g}{\partial\theta}(\theta)(\cos\theta,\sin\theta) + g(\theta)(-\sin\theta,\cos\theta)
      \end{align*}
      and
      \begin{align*}
        |\vec r'(\theta)|^2 = \left(\frac{\partial
          g}{\partial\theta}(\theta)\right)^2+g(\theta)^2.
      \end{align*}
      Hence
      \begin{align*}
        \mathrm{length}(\partial D) = \int_0^{2\pi}\sqrt{\left(\frac{\partial
          g}{\partial\theta}(\theta)\right)^2+g(\theta)^2}\,\dd \theta.
      \end{align*}
    }{\vspace{4cm} }

  \itm Let
    \begin{align*}
      (x(\theta),y(\theta)) = (g(\theta)\cos\theta, g(\theta)\sin\theta)
    \end{align*}
    be a parametrization of $\partial D$. Calculate the length of
    $\partial D$ again, but this time express the answer as an
    integral involving the derivatives of $x$ and $y$.

    \sol{
      The velocity is now $(x'(\theta),y'(\theta))$, and so 
      \begin{align*}
        \mathrm{length}(\partial D) = \int_0^{2\pi}\sqrt{x'(\theta)^2+y'(\theta)^2}\,\dd \theta.
      \end{align*}
    }{\vspace{4cm} }
    
    
  \itm Calculate the length of $\partial D$ for the case that
    $g(\theta) = 1-\cos\theta$. 

    \sol{
      Note that $g'(\theta) = \sin\theta$. Hence
      \begin{align*}
        \ell(a,b) &= \int_0^{2\pi}\sqrt{\left(\frac{\partial
              g}{\partial\theta}(\theta)\right)^2+g(\theta)^2}\,\dd \theta\\
        &= \int_0^{2\pi}\sqrt{\sin^2\theta+1-2\cos\theta+\cos^2\theta}\,\dd \theta\\
        &= \int_0^{2\pi}\sqrt{2(1-\cos\theta)}\,\dd \theta\\
        &= 8.
      \end{align*}
    }{\vspace{7cm} }

    \itm In the remainder of this problem we will prove an important
    theorem, called the {\em isoperimetric inequality} (this proof is due
    to E.\ Schmidt, from 1938): it states that the length of the boundary
    of any shape on the plane is at least equal to the square root of
    $4\pi$ times its area.

    Let $C$ be the unit circle.  Explain why
    \begin{align} \label{green}
      \mathrm{area}(D) +\pi &= \oint_{\partial D}(0,x)\cdot\dd
      \vec s + \oint_{C}(-y,0)\cdot\dd
      \vec s.
    \end{align}

    \sol{This follows from Green's Theorem.}{\vspace{4cm} }
   
 
  \itm
    Assume henceforth that $g(x) \leq 1$ and $g(0) = g(\pi) =
    1$. Show that
    \begin{align} \label{C-param}
      (x(\theta),w(\theta)) =
      \begin{cases}
        (g(\theta)\cos\theta,
        \sqrt{1-g(\theta)^2\cos^2\theta} & 0 \leq \theta \leq \pi \\
        (g(\theta)\cos\theta,
        -\sqrt{1-g(\theta)^2\cos^2\theta} & \pi \leq \theta  \leq 2\pi 
      \end{cases}
    \end{align} 
    is a parametrization of a unit circle.


    \sol{ Since $g(\theta) \leq 1$ then $w$ is real, and
      $x(\theta)^2+w(\theta)^2=1$. Hence all the points
      $(x(\theta),w(\theta))$ lie on the unit circle. Now, $x(0)=1$
      and $x(\pi)=-1$, since $g(0) = g(\pi) = 1$. Since $g$ is
      continuous $x$ is continuous, and so $x([0,\pi]) = [-1,1]$. It
      follows that $(x([0,\pi]),w([0,\pi]))$ is the upper half
      circle. By a similar argument $(x([\pi,2\pi]),w([\pi,2\pi]))$ is the
      lower half circle.
    }{\vspace{7cm} }
    
  \itm Let $(x(\theta),w(\theta))$ be the parametrization of the unit
    circle $C$ from~\eqref{C-param}. Again let
    \begin{align*}
      (x(\theta),y(\theta)) = (g(\theta)\cos\theta, g(\theta)\sin\theta)
    \end{align*}
    be a parametrization of $\partial D$. Using~\eqref{green}, show that
    \begin{align*}
      \mathrm{area}(D) +\pi = \oint_0^{2\pi}( x(\theta),-
      w(\theta))\cdot( y'(\theta),  x'(\theta))\,\dd\theta.
    \end{align*}
    
    \sol{
      \begin{align*}
        \mathrm{area}(D) +2\pi &= \oint_{\partial D}(0,x)\cdot\dd \vec s
        + \oint_{C}(-y,0)\cdot\dd
        \vec s\\
        &=
        \oint_0^{2\pi} x(\theta) y'(\theta)\,\dd \theta -\oint_0^{2\pi} w(\theta) x'(\theta)\,\dd\theta\\
        &=
        \oint_0^{2\pi} x(\theta) y'(\theta)- w(\theta) x'(\theta)\,\dd\theta\\
        &= \oint_0^{2\pi}( x(\theta),- w(\theta))\cdot( y'(\theta),
        y'(\theta))\,\dd\theta.
      \end{align*}    
    }{\vspace{7cm} }
    
    
  \itm Explain why it follows from the previous question that
    \begin{align*}
      \mathrm{area}(D) +\pi \leq \oint_0^{2\pi}\sqrt{(
        x(\theta)^2+ w(\theta)^2)\cdot(
        x'(\theta)^2+ y'(\theta)^2)}\,\dd\theta.
    \end{align*}
    
    \sol{The follows immediately from the
      Cauchy-Schwarz-Bunyakowski inequality for vectors in $\R^2$.  }{\vspace{4cm} }
    
    
  \itm Explain why
    \begin{align*}
      \mathrm{area}(D) +\pi \leq \mathrm{length}(\partial D).
    \end{align*}
    
    \sol{ Since $(x(\theta),w(\theta))$ is a parametrization of a
      unit circle, $x(\theta)^2+ w(\theta)^2=1$. The remaining
      integral is the length of $\partial D$.  }{\vspace{4cm} }
    
    \itm Recall the AMGM inequality: for $a,b>0$ it holds that
    $\sqrt{ab}\leq (a+b)/2$. Use this to show that
    \begin{align*}
      \sqrt{4\pi\cdot \mathrm{area}(D)} \leq \mathrm{length}(\partial D).
    \end{align*}
    For which shape are these two quantities equal?

    \sol{
      Since
      \begin{align*}
        \mathrm{area}(D) +\pi \leq \mathrm{length}(\partial D),
      \end{align*}
      \begin{align*}
        \frac{2\mathrm{area}(D) +2\pi}{2} \leq \mathrm{length}(\partial D).
      \end{align*}
      By the AMGM inequality
      \begin{align*}
        \sqrt{2\cdot\mathrm{area}(D)\cdot 2\pi} \leq \frac{2\mathrm{area}(D) +2\pi}{2},
      \end{align*}
      and so
    \begin{align*}
      \sqrt{4\pi\cdot \mathrm{area}(D)} \leq \mathrm{length}(\partial D).
    \end{align*}

    Equality is achieved for circles.
    }{\vspace{7cm} }


\end{document}

