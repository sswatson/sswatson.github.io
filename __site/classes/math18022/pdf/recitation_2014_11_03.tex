\documentclass[11pt]{article}

\usepackage[utf8x]{inputenc}
\usepackage[L7x]{fontenc}  
\usepackage[pdftex]{graphicx}
\usepackage{fourier}
\usepackage{hyperref}
\usepackage{amssymb}
\usepackage{amsmath}
\usepackage{wrapfig}
\usepackage{sectsty}
\usepackage{asymptote}
\usepackage{colonequals}
\usepackage{color}
\usepackage{calc}
\usepackage{etoolbox}

% COMMENT OUT THE SECOND LINE TO MAKE A HANDOUT WITHOUT SOLUTIONS
\newtoggle{solutions}
%\toggletrue{solutions}

\usepackage{amsthm}
\theoremstyle{definition}
\newtheorem{theorem}{Theorem}
\newtheorem{defn}{Definition}
\newtheorem{lemma}{Lemma}
\newtheorem{corollary}{Corollary}
\newtheorem{exercise}{Exercise}

\sectionfont{\large}
\sectionfont{\normalsize}

\def\Arg{\mathop{\rm Arg}\nolimits}
\def\Res{\mathop{\rm Res}}
\renewcommand\Im{\mathop{\rm Im}\nolimits}
\renewcommand\Re{\mathop{\rm Re}\nolimits}
\newcommand\Arctan{\mathop{\rm Arctan}\nolimits}
\newcommand{\R}{\mathbb{R}}
\newcommand{\C}{\mathbb{C}}
\newcommand{\N}{\mathbb{N}}
\newcommand{\Z}{\mathbb{Z}}
\renewcommand{\P}{\mathbb{P}}
\newcommand{\Chat}{\hat{\mathbb{C}}}
\newcommand{\UHP}{\mathbb{H}}
\DeclareMathOperator{\area}{area}
\DeclareMathOperator{\dist}{dist}
\DeclareMathOperator{\interior}{int}
\DeclareMathOperator{\id}{id}

\pagestyle{empty}

\textwidth = 6.5 in
\textheight = 9 in
\oddsidemargin = -0.125 in
\evensidemargin = 0.0 in
\topmargin = -0.2 in
\headheight = 0.0 in
\headsep = 0.2 in
\parskip = 0.2 in
\parindent = 0.0 in

\def\inv{^{-1}}

\newcounter{prob}
	\setcounter{prob}{1}

\newcounter{subprob}
	\setcounter{subprob}{1}

\newcommand\itm{\theprob.  \stepcounter{prob}\setcounter{subprob}{1}}
\newcommand\subitm{(\alph{subprob}) \refstepcounter{subprob}}

\newcommand\sol[2]{\iftoggle{solutions}{\textit{Solution}. #1}{#2}}
%\newcommand\sol[2]{#2}


\newcommand{\problem}[1]{
\makebox[0.2cm]{\textbf{\itm}}  \begin{minipage}[t]{\linewidth-0.75cm}
#1
\end{minipage}
}

\newcommand\twomatrix[4]{
\left(
\begin{array}{cc}
#1 & #2 \\
#3 & #4
\end{array}
\right)
}

\renewcommand\vec[1]{\mathbf{#1}}
\newcommand\pd[2]{\frac{\partial #1}{\partial #2}}

\begin{document}
\thispagestyle{empty}

\begin{center}
  18.022 Recitation Handout \iftoggle{solutions}{(with solutions)}{} \\
  3 November 2014 
\end{center}


\itm Evaluate $\displaystyle{\int_{0}^{1}\!\int_{0}^{y^2} x^2 y \,dx\,dy}$ and sketch the region of integration in $\R^2$ indicated by the limits of integration. 

\sol{
The domain of integration is shown below. 
\begin{center}
  \includegraphics{figures/region}
\end{center}
Evaluating the integral, we find 
  \begin{align*}
    \int_{0}^{1}\!\int_{0}^{y^2} x^2 y \,dx\,dy &= \int_{0}^{1} \left[\frac{x^3}{3}y\right]_{0}^{y^2}\,dy  \\
    &= \int_0^1 \frac{y^7}{3}\,dy \\
    &= \boxed{1/24}. \qedhere
  \end{align*}
}{\vfill}


\itm Evaluate $\int_{0}^\pi \!\int_{y}^{\pi}\frac{\sin x}{x} \,dx\,dy$.

\sol{We reverse the order of integration. The region of integration is a triangle with vertices at $(0,0)$, $(\pi,0)$, and $(\pi,\pi)$. So the integral is equal to 
\[
\int_{0}^{\pi}\!\int_{0}^{x}\frac{\sin x}{x} \,dy\,dx = \int_0^\pi \left[y\right]_{0}^x \sin(x)/x \,dx = \int_0^\pi \sin x \,dx = \boxed{2}.  \qedhere
\] 
}{\vfill}

\iftoggle{solutions}{}{\newpage} 

\itm (Putnam exam '89) Evaluate
$\displaystyle{\int_0^a\int_0^b e^{{\rm max}\{b^2x^2, a^2y^2\}}\,dy\,dx}$
where $a$ and $b$ are positive.
\sol{ Omitted. }{\vfill}

\itm (Fun/Challenge, based on 5.2.29 in \textit{Colley}) Define a function $f(x,y)$ on $[0,1] \times [0,2]$ by
\[
f(x,y) = \left\{ \begin{array}{cl}
1 & \text{ if }x\text{ is rational} \\
0 & \text{ if }x\text{ is irrational and }y \leq 1 \\
2 & \text{ if }x\text{ is irrational and }y > 1. 
\end{array} 
\right.
\]
Show that the iterated Riemann integral $\int_0^1\int_0^2 f(x,y)\,dy\,dx$ exists, and find its value. Show that the iterated Riemann integral $\int_0^2\int_0^1 f(x,y)\,dx\,dy$ does not exist. 

\sol{To show that $\int_0^1\int_0^2 f(x,y)\,dy\,dx$ exists, we first calculate $\int_0^2 f(x,y)\,dy$. If $x$ is rational, then this integral is $\int_0^21\,dy = 2$. If $x$ is irrational, then this integral reduces to $\int_1^2 2\,dy =2$. Therefore, regardless of the value of $x$, the inner integral $\int_0^2 f(x,y)\,dy$ is equal to 2. Therefore, $\int_0^1\int_0^2 f(x,y)\,dy\,dx=\int_0^12\,dx=2$. 

On the other hand, there is no value of $y$ for which the integral $\int_0^1 f(x,y)\,dx$ exists. To see this, note that (for $y\leq 1$, say) the upper and lower Riemann sums are equal to 1 and 0, respectively. For if $0=x_0<x_1<\ldots<x_n = 1$ is some partition of $[0,1]$, then the sum
\[
\sum_{k=0}^{n-1} f(x_k^*)(x_{k+1}-x_k)
\]
can be made as large as 1 by choosing each $x_k^*\in(x_{k},x_{k+1})$ to be
rational and as small as 0 by choosing each $x_k^*\in(x_{k},x_{k+1})$ to be
irrational. 

Since the inner integral doesn't exist, the iterated integral doesn't exist
either.  }{\vfill}



\end{document}

