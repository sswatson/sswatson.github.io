\documentclass[11pt]{article}

\usepackage[utf8x]{inputenc}
\usepackage[L7x]{fontenc}  
\usepackage[pdftex]{graphicx}
\usepackage{fourier}
\usepackage{amssymb}
\usepackage{amsmath}
\usepackage{wrapfig}
\usepackage{sectsty}
\usepackage{asymptote}
\usepackage{colonequals}
\usepackage{color}
\usepackage{calc}
\usepackage{etoolbox}

% COMMENT OUT THE SECOND LINE TO MAKE A HANDOUT WITHOUT SOLUTIONS
\newtoggle{solutions}
%\toggletrue{solutions}

\usepackage{amsthm}
\theoremstyle{definition}
\newtheorem{theorem}{Theorem}
\newtheorem{defn}{Definition}
\newtheorem{lemma}{Lemma}
\newtheorem{corollary}{Corollary}
\newtheorem{exercise}{Exercise}

\sectionfont{\large}
\sectionfont{\normalsize}

\def\Arg{\mathop{\rm Arg}\nolimits}
\def\Res{\mathop{\rm Res}}
\renewcommand\Im{\mathop{\rm Im}\nolimits}
\renewcommand\Re{\mathop{\rm Re}\nolimits}
\newcommand\Arctan{\mathop{\rm Arctan}\nolimits}
\newcommand{\R}{\mathbb{R}}
\newcommand{\C}{\mathbb{C}}
\newcommand{\N}{\mathbb{N}}
\newcommand{\Z}{\mathbb{Z}}
\renewcommand{\P}{\mathbb{P}}
\newcommand{\Chat}{\hat{\mathbb{C}}}
\newcommand{\UHP}{\mathbb{H}}
\DeclareMathOperator{\area}{area}
\DeclareMathOperator{\dist}{dist}
\DeclareMathOperator{\interior}{int}
\DeclareMathOperator{\id}{id}

\pagestyle{empty}

\textwidth = 6.5 in
\textheight = 9 in
\oddsidemargin = -0.125 in
\evensidemargin = 0.0 in
\topmargin = -0.2 in
\headheight = 0.0 in
\headsep = 0.2 in
\parskip = 0.2 in
\parindent = 0.0 in

\def\inv{^{-1}}

\newcounter{prob}
	\setcounter{prob}{1}

\newcounter{subprob}
	\setcounter{subprob}{1}

\newcommand\itm{\theprob.  \stepcounter{prob}\setcounter{subprob}{1}}
\newcommand\subitm{(\alph{subprob}) \refstepcounter{subprob}}

\newcommand\sol[2]{\iftoggle{solutions}{\textit{Solution}. #1}{#2}}
%\newcommand\sol[2]{#2}


\newcommand{\problem}[1]{
\makebox[0.2cm]{\textbf{\itm}}  \begin{minipage}[t]{\linewidth-0.75cm}
#1
\end{minipage}
}

\newcommand\twomatrix[4]{
\left(
\begin{array}{cc}
#1 & #2 \\
#3 & #4
\end{array}
\right)
}

\renewcommand\vec[1]{\mathbf{#1}}
\newcommand\pd[2]{\frac{\partial #1}{\partial #2}}

\begin{document}
\thispagestyle{empty}

\begin{center}
  18.022 Recitation Handout \iftoggle{solutions}{(with solutions)}{} \\
  8 October 2014 
\end{center}
\itm Sketch the image of the path $\vec{x}(t)=(\cos t, e^t)$. 


\sol{
See the graph below. There is actually a lot of oscillating that happens
near the origin, but you can't see it in the graph because $e^t$ converges
to zero rapidly as $t<0$ goes to $-\infty$. 
  \begin{center}
    \includegraphics{figures/parametric5}
  \end{center}
}{
\vfill
}

\itm (3.1.25 in \textit{Colley}) A malfunctioning rocket is traveling according to a path $\vec{x}(t)=(e^{2t},3t^3-2t, t-1/t)$ in the hope of reaching a repair station at the point $(7e^4,35,5)$. (Here $t$ represents time in minutes and spatial coordinates are measured in miles). At $t=2$, the rocket's engines suddenly cease. Will the rocket coast into the repair station? 

\sol{
  The rocket's direction vector at time $t$ is
  $\vec{x}'(t)=(2e^{2t},9t^2-2, 1+1/t^2)$, which at the time the engine
  stops is $(2e^4,34,5/4)$. To check whether the rocket reaches the station
  at the point $P=(7e^4,35,5)$, we just have to check whether the vector
  $P-\vec{x}(2)=(7e^4,35,5)-(e^{4},20, 3/2)=(6e^4,15,7/2)$ is parallel to
  the velocity vector $(2e^4,34,5/4)$. The ratio of $x$-components is not
  the same as the ratio of $y$-components, so there is no scalar $\lambda$
  for which $P-\vec{x}(2) = \vec{x}'(2)$.
}
{\vfill}

\iftoggle{solutions}{}{\newpage}

\itm (3.2.7 in \textit{Colley}) Calculate total length of the curve given by $(a\cos^3t,a\sin^3 t)$, where $a$ is a positive constant. This is the shape you get when you roll a circle of radius $a/4$ around inside a circle of radius $a$ and track the trajectory of a point on the smaller circle (see below). 


\iftoggle{solutions}{\begin{wrapfigure}[6]{r}{4cm}
      \includegraphics{figures/astroid}
    \end{wrapfigure}}{}
\sol{ 
We first consider $t$ ranging over $[0,\pi/2]$. We get an arclength of
\begin{align*}
\int_{0}^{\pi/2} \sqrt{(-3a\cos^2t \sin t)^2+(3a\sin^2t\cos t)^2}\,dt &= 
3a \int_{0}^{\pi/2} \sqrt{\sin^2t \cos^2 t (\cos^2 t + \sin^2 t)} \\ &=  
3a \int_{0}^{\pi/2} \sin t \cos t \,dt. 
\end{align*}
In the last line, we have used the fact that $\sin t \cos t \geq 0$. This is
why it was useful to assume $t\in [0,\pi/2]$.\footnote{If $x<0$, it is not
  correct to say that $\sqrt{x^2} = x$.}
}{}


\iftoggle{solutions}{
Writing $\sin t \cos t=\frac{1}{2}
\sin 2t,$ we integrate to get $3a/2$ for the arclength of one quarter of
the hypocycloid. Therefore, the total arclength is $\boxed{6a}$. 
}{
\begin{flushright} \includegraphics[width=3cm]{figures/astroid}
\end{flushright}

\vfill 
}


\itm Explain why the arclength of $\sin(1/x)$ over $x\in [0,1]$ does not
exist (no calculation necessary). 

\sol{The arclength is infinite because the graph oscillates infinitely many
times between $-1$ and $1$. Each of these oscillations contributes an
arclength of at least 2, and $2+2+\ldots = +\infty$.}{\vspace{1.0cm}} 

(b) Does the arclength of $x\sin(1/x)$ over $x\in [0,1]$ exist? 

\sol{The arclength is infinite for this example as well. Consider the
  arclength contributed by the portions of the graph above the intervals
  $I_n=(\frac{1}{\pi (2n+1)},\frac{1}{2\pi n})$ for $n$ ranging over the
  positive integers. These are the intervals where the function is
  positive. Note that the arclength of the graph above $I_n$ is at least
  twice the maximum of the function on the interval, which is at least
  $\frac{2}{\pi/2+2\pi n}$. Summing this expression over $n$ gives
  $+\infty$, by comparison with $\sum\frac1n$.}{\vspace{2cm}}

(c) Does the arclength of $x^2 \sin(1/x)$ over $x\in [0,1]$ exist? 

\sol{The analysis in the previous problem suggests that the arclength is
  finite. That analysis can be made rigorous for this problem too, but we
  need to upper bound the arclength corresponding to each interval, and
  this is more difficult than a lower bound.

  Alternatively, we can calculate the arclength as 
\begin{align*}
\int_{0}^1 \sqrt{1+f'(x)^2}\,dx &= \int_0^1
\sqrt{1+(2x\sin(1/x)+x^2(-1/x^2)\cos(1/x))^2}\,dx \\ &= \int_0^1
  \sqrt{1+(2x\sin(1/x)-\cos(1/x))^2}\,dx. 
\end{align*}
This is an integral of a bounded function over a bounded interval and
therefore is finite. 
}{\vspace{3cm}}

(d) (Fun/Challenge) Determine the values of $m$ and $n$ for which $x^m\sin(x^n)$ has finite arc length over $x\in[0,1]$.

\sol{Left as an exercise.}{\vfill}

\vspace{0.5cm} 

\end{document}

