\documentclass[11pt]{article}

\usepackage[utf8x]{inputenc}
\usepackage[T1]{fontenc}  
\usepackage[pdftex]{graphicx}
\usepackage{hyperref}
\usepackage{amssymb}
\usepackage{amsmath}
\usepackage{wrapfig}
\usepackage{sectsty}
\usepackage{asymptote}
\usepackage{colonequals}
\usepackage{color}
\usepackage{calc}
\usepackage{etoolbox}

\allowdisplaybreaks

% COMMENT OUT THE SECOND LINE TO MAKE A HANDOUT WITHOUT SOLUTIONS
\newtoggle{solutions}
\toggletrue{solutions}

\usepackage{amsthm}
\theoremstyle{definition}
\newtheorem{theorem}{Theorem}
\newtheorem{defn}{Definition}
\newtheorem{lemma}{Lemma}
\newtheorem{corollary}{Corollary}
\newtheorem{exercise}{Exercise}

\usepackage{palatino} 
\usepackage{pxfonts} 

\sectionfont{\large}
\sectionfont{\normalsize}

\def\Arg{\mathop{\rm Arg}\nolimits}
\def\Res{\mathop{\rm Res}}
\renewcommand\Im{\mathop{\rm Im}\nolimits}
\renewcommand\Re{\mathop{\rm Re}\nolimits}
\newcommand\Arctan{\mathop{\rm Arctan}\nolimits}
\newcommand{\R}{\mathbb{R}}
\newcommand{\C}{\mathbb{C}}
\newcommand{\N}{\mathbb{N}}
\newcommand{\Z}{\mathbb{Z}}
\renewcommand{\P}{\mathbb{P}}
\newcommand{\Chat}{\hat{\mathbb{C}}}
\newcommand{\UHP}{\mathbb{H}}
\DeclareMathOperator{\area}{area}
\DeclareMathOperator{\dist}{dist}
\DeclareMathOperator{\interior}{int}
\DeclareMathOperator{\id}{id}

\pagestyle{empty}

\textwidth = 6.5 in
\textheight = 9 in
\oddsidemargin = -0.125 in
\evensidemargin = 0.0 in
\topmargin = -0.2 in
\headheight = 0.0 in
\headsep = 0.2 in
\parskip = 0.2 in
\parindent = 0.0 in

\def\inv{^{-1}}

\newcounter{prob}
	\setcounter{prob}{1}

\newcounter{subprob}
	\setcounter{subprob}{1}

\newcommand\itm{\theprob.  \stepcounter{prob}\setcounter{subprob}{1}}
\newcommand\subitm{(\alph{subprob}) \refstepcounter{subprob}}

\newcommand\sol[2]{\iftoggle{solutions}{\begin{proof}[Solution] #1\end{proof}}{#2}}
%\newcommand\sol[2]{\iftoggle{solutions}{\textit{Solution}. #1}{#2}}
%\newcommand\sol[2]{#2}


\newcommand{\problem}[1]{
\makebox[0.2cm]{\textbf{\itm}}  \begin{minipage}[t]{\linewidth-0.75cm}
#1
\end{minipage}
}
\newcommand\twomatrix[4]{
\left(
\begin{array}{cc}
#1 & #2 \\
#3 & #4
\end{array}
\right)
}

\newcommand\twomatrixdet[4]{
\left|
\begin{array}{cc}
#1 & #2 \\
#3 & #4
\end{array}
\right|
}


\renewcommand\vec[1]{\mathbf{#1}}
\newcommand\pd[2]{\frac{\partial #1}{\partial #2}}

\begin{document}
\thispagestyle{empty}

\begin{center}
  18.022 Recitation Handout \iftoggle{solutions}{(with solutions)}{} \\
  13 November 2014 \\

\end{center}


% $\displaystyle{\int_{}^{}\int_{}^{}\,dx\,dy}$


\itm (5.5.10 in \textit{Colley}) Evaluate the integral $\displaystyle{\int_{0}^{2}\!\int_{x/2}^{x/2+1} x^5 (2y - x)e^{(2y-x)^2} \,dy\,dx}$ by making the substitution $u = x$ and $v=2y-x$. 

\sol{
Substitution shows that the limits of integration become $u \in [0,2]$ and $v\in [0,2]$. We calculate 
\[
\left|\frac{\partial(x,y)}{\partial(u,v)}\right| = \left|
\begin{array}{cc}
1 & 1/2 \\
0 & 1/2
\end{array}
\right|=1/2,
\]
so the formula for change of variables gives 
  \begin{align*}
    \int_{0}^{2}\!\int_{0}^{2} u^5 v e^{v^2} \left|\frac{\partial(x,y)}{\partial(u,v)}\right| \,dv\,du &=
    \int_{0}^{2} \frac{1}{2}\left[\frac12 u^5 e^{v^2}\right]_{0}^{2} \, du \\
    &= \int_{0}^{2} \frac14 u^5 (e^4-1) \, du 
    = \boxed{\frac{8}{3}(e^4-1)}. \qedhere
  \end{align*}
}{\vfill}
\itm Let $D$ be a parallelogram with vertices $(0,0)$, $(1,0)$, $(1,1)$, and $(2,1)$. Calculate $\iint_D 1\,dA$ in two ways:

(a) Find $\iint_D 1\,dA$ without using calculus. 

(b) Find $\iint_D 1\,dA$ using the change of variables $u = 2x - 2y$ and $v=2y$. 

\sol{
(a) The integral in question is the area of the parallelogram, which is $(\text{base})(\text{height})=1\times 1 = \boxed{1}$. (b) To calculate this area using the suggested change of variables, we note that the linear transformation sends the four vertices of $D$ to the vertices of a square $[0,2]\times[0,2]$. Since the transformation is linear, it sends $D$ to $[0,2]\times[0,2]$. We calculate 
\begin{align*}
  \iint_D 1\,dA &= \int_0^2\!\int_0^2 1 \twomatrixdet{\partial x/\partial u}{\partial y/\partial u}{\partial x/\partial v}{\partial y/ \partial v} du \,dv\\
&= \int_0^2\!\int_0^2 1 \twomatrixdet{\frac12}{0}{\frac12}{1/2} du \,dv\\
&= 4(1/2)^2 = \boxed{1}. \qedhere
\end{align*}
}{\vfill} 

\iftoggle{solutions}{}{\newpage} 


\itm (5.5.30 in \textit{Colley}) Find the volume of the solid that is bounded by the paraboloid $z = 9 - x^2 - y^2$, the $xy$-plane, and the cylinder $x^2 + y^2 = 4$. 

\sol{
  To find the volume of the region, we integrate 1 over the region. We use cylindrical coordinates:
\begin{equation*}
  \text{volume} = \int_{0}^{2\pi} \!\int_0^2 \int_{0}^{9-r^2} 1 \,r\, dz \,dr\,d\theta =   (2\pi) \!\int_0^2(9-r^2)r\,dr. = \boxed{28\pi}. \qedhere
\end{equation*}
}{\vfill} 

\itm (5.5.29 in \textit{Colley}) Find the volume of the region $W$ that
represents the intersection of the solid cylinder $x^2+y^2 \leq 1$ and the
solid ellipsoid $2(x^2+y^2)+z^2\leq 10.$ 

\sol{We integrate in cylindrical coordinates: 
\[
\int_{0}^{1}\int_{0}^{2\pi}\int_{-\sqrt{10-2r^2}}^{\sqrt{10-2r^2}}\,r\,dz\,dr\,
d\theta = 2\pi \int_0^12r\sqrt{10-2r^2}\,dr = 
\boxed{\frac{4}{3} \sqrt{2} \left(5 \sqrt{5}-8\right) \pi}.\qedhere
\]}{\vfill}




\end{document}

